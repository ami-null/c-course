\documentclass[12pt, aspectratio=169]{beamer}

\input{../header}
\usepackage{minted}
\renewcommand{\theFancyVerbLine}{\ttfamily{\small\oldstylenums{\arabic{FancyVerbLine}}}}
\setminted{
    frame=none,
    bgcolor=gray!20,
    % linenos, 
    % numbers=left,
    % numbersep=5pt
    breaklines=true,
    fontsize=\fontsize{12pt}{13pt}\selectfont
}

\input{../header_tikz}

\title{Conditional Execution and Loops in C}
\author{Md. Aminul Islam Shazid}
\date{}

\begin{document}

    {
		\setbeamertemplate{footline}{}    % NO FOOTLINE FOR THESE TWO FRAMES
		\addtocounter{framenumber}{-2}    % not counting the title page and the outline in frame numbers

		\begin{frame}
			\titlepage
		\end{frame}

		\begin{frame}{Outline}
            % \vfill
            \small
			\tableofcontents[subsectionstyle=hide]
            % \vfill
		\end{frame}
	}

    \section{Conditional Execution}

    \begin{frame}{Conditional Execution in C}
        \begin{itemize}
            \item \textbf{Conditional execution} allows a program to take different actions based on certain conditions
            \item Conditions are expressed using \textttbg{if}, \textttbg{else}, and \textttbg{else if} statements
            \item Condition expressions must evaluate to \textttbg{true (non-zero)} or \textttbg{false (zero)}
        \end{itemize}
    \end{frame}


    \begin{frame}[fragile]{\texttt{if} Statement}
        \textbf{Syntax:}
        \begin{minted}{c}
if(condition){
    // statements
}
        \end{minted}

        \textbf{Example:}
        \inputminted[
            linenos, 
            % firstline=1,
            % lastline=7,
            % firstnumber=1
        ]{c}{../code-examples/06_01_check-positive.c}
    \end{frame}


    \begin{frame}[fragile]{\texttt{if-else} Statement}
        \begin{minted}{c}
if(condition){
    // commands to execute if true
} else{
    // commands to execute if false
}
        \end{minted}

    \end{frame}


    \begin{frame}[fragile]{Example: \texttt{if-else}}
        \inputminted[
            linenos, 
        ]{c}{../code-examples/06_02_check-voter.c}
    \end{frame}


    \begin{frame}[fragile]{\texttt{else if} Ladder}
        \begin{minted}{c}
if(condition1){
    ...
} else if(condition2) {
    ...
} else{
    ...
}
        \end{minted}
    \end{frame}


    \begin{frame}[fragile]{Example: \texttt{else if} Ladder}
        \inputminted[
            linenos, 
        ]{c}{../code-examples/06_03_letter-grade.c}
    \end{frame}


    \begin{frame}{Boolean Algebra in \texttt{if} Statements}
        \begin{itemize}
            \item There can be multiple conditions
            \item Need to perform Boolean algebra on these conditions, because \textttbg{if} statement expects only a single value
            \item Boolean operations on multiple conditions evaluate to a single value (true or false)
            \item Boolean operators:
            \begin{itemize}
                \item AND, \textttbg{\&\&}: Code runs only if all conditions are true
                \item OR \textttbg{||}: Code runs if at least one condition is true
                \item NOT \textttbg{!}: Negates a condition (flips true to false, and vice-versa)
            \end{itemize}
        \end{itemize}
    \end{frame}


    \begin{frame}[fragile]{Example: Loan Eligibility (AND operator)}
        \inputminted[
            linenos, 
        ]{c}{../code-examples/06_04_loan_and-operator.c}
    \end{frame}


    \begin{frame}[fragile]{Example: Age Check (AND operator)}
        \inputminted[
            linenos, 
        ]{c}{../code-examples/06_05_age_and-operator.c}
    \end{frame}


    \begin{frame}[fragile]{Example: Sports Eligibility (OR operator)}
        \inputminted[
            linenos, 
        ]{c}{../code-examples/06_06_sports-eligibility_or-operator.c}
    \end{frame}



    \section{Nested \textttbg{if} Statements}

    \begin{frame}[fragile]{Nested \texttt{if} Statements}
        Sometimes, it necessary to put an \textttbg{if} statement inside another. This is called nested statements. Can have as many levels of nesting as necessary.

        \begin{minted}{c}
if(cond1){
    // code that's executed if cond1 is true
    if(cond2){
        // executed if both cond1 and cond2 are true
    } else{
        // executed if cond1 is true and cond2 is false
    }
    // code that's executed if cond1 is true
}
        \end{minted}
    \end{frame}


    \begin{frame}[fragile]{Example: Loan Eligibility (revisited)}
        \inputminted[
            linenos, 
        ]{c}{../code-examples/06_07_loan_nested-if.c}
    \end{frame}


    \section{Loop}

    \begin{frame}{Loops in C}
        \begin{itemize}
            \item Loops are used to execute a block of code repeatedly.
            \item Types of loops in C:
            \begin{itemize}
                \item \textttbg{for} loop: when number of iterations is known
                \item \textttbg{while} loop: when condition is checked before each iteration
                \item \textttbg{do-while} loop: condition checked \textit{after} executing loop body
            \end{itemize}
        \end{itemize}

        In do-while loop, the body of the loop is always executed at least once.
    \end{frame}


    \begin{frame}[fragile]{\texttt{for} Loop}
        \begin{minted}{c}
for(initialization; condition; update){
    // statements
}
        \end{minted}

        The elements (initialization, condition and update) inside the for keyword, can be ommitted. For example,
        \begin{itemize}
            \item Initialization can be performed before the \textttbg{for} keyword
            \item Condition and update can moved inside the loop body
            \item \textttbg{for(;;)\{...\}} creates an infinite loop
        \end{itemize}
    \end{frame}


    \begin{frame}[fragile]{Example: \texttt{for} Loop}
        \inputminted[
            linenos, 
        ]{c}{../code-examples/06_08_for_print-1-to-5.c}
    \end{frame}


    \begin{frame}[fragile]{Example: Another Way to Construct \texttt{for} Loops}
        \inputminted[
            linenos, 
        ]{c}{../code-examples/06_09_for_empty-init.c}
    \end{frame}


    \begin{frame}[fragile]{Example: Sum Odd Integers (\texttt{if} inside \texttt{for})}
        \inputminted[
            linenos, 
        ]{c}{../code-examples/06_15_for-if_sum-odd-integers.c}
    \end{frame}


    \begin{frame}[fragile]{Example: Sum Odd Integers (No \texttt{if} statement)}
        \inputminted[
            linenos, 
        ]{c}{../code-examples/06_16_for_sum-odd-integers.c}
    \end{frame}


    \begin{frame}[fragile]{\texttt{while} Loop}
        \textbf{Syntax:}
        \begin{minted}{c}
while(condition){
    // statements
}
        \end{minted}

        \textbf{Example:}
        \inputminted[
            linenos, 
        ]{c}{../code-examples/06_10_while_print-1-to-5.c}
    \end{frame}


    \begin{frame}{Example: Greatest Common Divisor (GCD)}
        \begin{itemize}
            \item The GCD of two integers $a$ and $b$ is $c$ if both $a$ and $b$ are divisible by $c$
            \item First, assume that the smaller number is the GCD
            \item Then check if both $a$ and $b$ are divisible by the assumed GCD. If not, then decrement the assumed value by 1
            \item Keep repeating this process until both $a$ and $b$ are found to be divisible
        \end{itemize}
    \end{frame}


    \begin{frame}{Example: GCD (cont.)}
        \inputminted[
            linenos, 
            firstline=1,
            lastline=11,
            firstnumber=1
        ]{c}{../code-examples/06_11_while_gcd.c}

        \textit{Continued in the next page}
    \end{frame}


    \begin{frame}{Example: GCD (cont.)}
        \inputminted[
            linenos, 
            firstline=12,
            % lastline=11,
            firstnumber=12
        ]{c}{../code-examples/06_11_while_gcd.c}
    \end{frame}


    \begin{frame}[fragile]{\texttt{do-while} Loop}
        \textbf{Syntax:}
        \begin{minted}{c}
do{
    // statements
} while(condition);    // don't forget this semicolon
        \end{minted}

        \textbf{Example:}
        \inputminted[
            linenos, 
        ]{c}{../code-examples/06_12_do-while.c}
    \end{frame}


    \begin{frame}{\texttt{while} vs \texttt{do-while} Loop}
        \begin{itemize}
            \item \textttbg{while}: condition checked \textit{before} loop body
            \item \textttbg{do-while}: condition checked \textit{after} running the first iteration of the loop, so the loop runs at least once
        \end{itemize}
    \end{frame}


    \begin{frame}{Flowchart: While vs Do-While}
        \begin{columns}
        \begin{column}{0.5\textwidth}
            \begin{tikzpicture}
                \node (start) [startstop] {Start};
                \node (cond) [decision, below of=start, yshift=-1cm] {Condition};
                \node (command) [process, below of=cond, yshift=-1cm] {Perform commands};
                \node (stop) [startstop, below of=command, yshift=-1cm] {Stop};

                \draw [arrow] (start) -- (cond);
                \draw [arrow] (cond) -- (command) node[midway, right] {True};
                \draw [arrow] (command.west) -- ++(-1, 0) |- (cond);
                \draw [arrow] (cond.east) -- node[midway, above] {False} ++(1.5, 0) |- (stop);
            \end{tikzpicture}
        \end{column}
        \begin{column}{0.5\textwidth}
            \begin{tikzpicture}
                \node (start) [startstop] {Start};
                \node (command) [process, below of=start, yshift=-1cm] {Perform commands};
                \node (cond) [decision, below of=command, yshift=-1cm] {Condition};
                \node (stop) [startstop, below of=cond, yshift=-1cm] {Stop};

                \draw [arrow] (start) -- (command);
                \draw [arrow] (command) -- (cond);
                \draw [arrow] (cond.east) -- ++(1.5, 0) node[midway, above] {True} |- (command.east);
                \draw [arrow] (cond) -- (stop) node[midway, right] {False};
            \end{tikzpicture}
        \end{column}
        \end{columns}
    \end{frame}


    \begin{frame}{Example: Input Validation Using \texttt{do-while}}
        \inputminted[
            linenos, 
        ]{c}{../code-examples/06_17_do-while_input-validation.c}
    \end{frame}


    \section{\texttt{break} and \texttt{continue}}

    \begin{frame}{The \texttt{break} Statement}
        \begin{itemize}
            \item The break statement immediately terminates the loop or switch statement in which it is encountered
            \item Control of the program then transfers to the statement immediately following the loop or switch
            \item It is commonly used to exit a loop prematurely based on a certain condition
        \end{itemize}
    \end{frame}


    \begin{frame}[fragile]{Example: \texttt{break}}
        \inputminted[
            linenos, 
        ]{c}{../code-examples/06_13_break.c}
    \end{frame}


    \begin{frame}{The \texttt{continue} Statement}
        \begin{itemize}
            \item The continue statement skips the remaining statements in the current iteration of a loop and proceeds to the next iteration
            \item It is used when you want to bypass certain parts of the loop's body for specific conditions without exiting the entire loop
        \end{itemize}
    \end{frame}


    \begin{frame}[fragile]{Example: \texttt{continue}}
        \inputminted[
            linenos, 
        ]{c}{../code-examples/06_14_continue.c}
    \end{frame}


    \section{Nested Loops}

    \begin{frame}{Nested Loops}
        \begin{itemize}
            \item A \textbf{nested loop} means one loop inside another loop
            \item The \textbf{inner loop} executes completely for every single iteration of the \textbf{outer loop}
            \item Commonly used for:
            \begin{itemize}
                \item Working with 2D data (like matrices)
                \item Generating patterns
                \item Performing repeated comparisons or calculations
            \end{itemize}
            \item One can nest as many loops as necessary, but nesting more than two or three loops can lead to confusing or hard to understand codes
        \end{itemize}
    \end{frame}


    \begin{frame}[fragile]{Nested Loops: Basic Syntax}
        \begin{minted}{c}
for(initialization; condition; update){
    for(initialization; condition; update){
        // inner loop body
    }
    // outer loop body
}
        \end{minted}

        \begin{itemize}
            \item You can nest \textttbg{while} inside \textttbg{for}, or any combination of loop types
            \item Be careful with initialization and loop conditions to avoid infinite loops
        \end{itemize}
    \end{frame}


    \begin{frame}{Example: (Non-nested, single loop) Multiplication table of 3}
        \inputminted[
            linenos, 
        ]{c}{../code-examples/06_18_single-loop_multiplication-table.c}
    \end{frame}


    \begin{frame}{Example: (Nested loops) Multiplication tables of 1, 2 and 3}
        \inputminted[
            linenos, 
        ]{c}{../code-examples/06_19_nested-loops_multiplication-table.c}
    \end{frame}


    \begin{frame}{Example: Triangle Pattern with *}
        \inputminted[
            linenos, 
        ]{c}{../code-examples/06_20_nested-loops_triangle.c}
    \end{frame}


    \begin{frame}{Example: Number Triangle}
        \inputminted[
            linenos, 
        ]{c}{../code-examples/06_21_number-triangle.c}
    \end{frame}


    \begin{frame}{Example: Number Pyramid}
        \inputminted[
            linenos, 
        ]{c}{../code-examples/06_22_number-pyramid.c}
    \end{frame}


    \section{The \textttbg{switch} Statement}

    \begin{frame}{The \texttt{switch} Statement}
        \begin{itemize}
            \item The \textttbg{switch} statement allows multi-way branching based on the value of an expression
            \item It is an alternative to long chains of \textttbg{if-else-if} statements
            \item It compares the given expression against multiple constant values given by (\textttbg{case} labels)
            \item The \textttbg{default} case handles unexpected or unmatched values
        \end{itemize}
    \end{frame}


    \begin{frame}[fragile]{\texttt{switch} Statement General Syntax}
        \begin{minted}{c}
switch(expression){
    case value1:
        // statements
        break;
    case value2:
        // statements
        break;
    // ...
    // ...
    default:
        // statements (optional)
}
        \end{minted}
    \end{frame}


    \begin{frame}{Why \texttt{break} is Necessary}
        \begin{itemize}
            \item Without \textttbg{break}, execution ``falls through" to the next \textttbg{case}
            \item This means all subsequent cases are executed until a \textttbg{break} or the end of the switch
            \item To prevent this, use \textttbg{break} at the end of each case
        \end{itemize}
    \end{frame}


    \begin{frame}[fragile]{Example: Fall-Through Behavior of \texttt{switch}}
        \begin{minted}{c}
int x = 2;
switch(x){
    case 1: 
        printf("A ");
    case 2: 
        printf("B ");
    case 3: 
        printf("C ");
}
// Output: B C
        \end{minted}
    \end{frame}


    \begin{frame}{Example: Even-Odd}
        \inputminted[
            linenos, 
        ]{c}{../code-examples/06_23_switch_even-odd.c}
    \end{frame}


    \section{Exercise}

    % \begin{frame}[fragile, allowframebreaks=0.75]{Exercise}
    \begin{frame}[fragile, allowframebreaks]{Exercise}
        Write C programs:
        \begin{enumerate}
            \item To check whether a number (user input) is positive or negative or zero
            \item To check whether a year (user input) is a leap year
            \item To check whether an integer is even or odd
            \item To find the number of real-valued solution(s) to a quadratic equation, (\(ax^2+bx+c=0\)). Take \textttbg{a}, \textttbg{b} and \textttbg{c} as user inputs. Then calculate the value of the discriminant, then show the appropriate output
            \item To print the first n (user input) natural numbers using a \textttbg{for} loop. And another program to do the same using a \textttbg{while} loop
            \item To compute the sum of numbers from 1 to n using a \textttbg{for} loop. And another program to do the same using a \textttbg{while} loop
            \item To find the factorial of an integer (user input)
            \item To print the first n (user input) terms of the fibonacci series
            \item To print the first n (user input) terms of the following arithmetic progression sequence: 1, 4, 7, 10, 13\dots
            \item To repeatedly take user input and print its square, until a negative number is entered (use while loop)
            \item To repeatedly take user input as exam marks and print the corresponding letter grade, until a negative number is entered (use \textttbg{while} loop and \textttbg{if} statement)
            \item To find the GCD of two integers using the Euclidean algorithm
            \item To find the LCM of two integers
            \item To repeatedly take user input and print its square, until a negative number is entered (use \textttbg{do-while} loop)
            \item To repeatedly take user input as exam marks and print the corresponding letter grade, until a negative number is entered (use \textttbg{do-while} loop)
            \item To print the sum of the first n (user input) terms of the following arithmetic progression sequence: 1 + 4 + 7 + 10 + 13\dots
            \item To print the first n (user input) terms of the following sequence: 1, 2, 4, 7, 11, 16\dots
            \item To print the sum of the first n (user input) terms of the following series: 1 + 2 + 4 + 7 + 11 + 16\dots
            \item To find all the prime numbers within a given range. The start and end integers of the range shall be user input
            \item To print a right aligned triangle pattern with *, sample output:
            \begin{verbatim}
        *
      * *
    * * *
  * * * *
* * * * *
            \end{verbatim}

            \framebreak

            \item To generate a multiplication table up to 5×5 in grid format, sample output:
            \begin{verbatim}
  1   2   3   4   5
  2   4   6   8  10
  3   6   9  12  15
  4   8  12  16  20
  5  10  15  20  25
            \end{verbatim}

            \framebreak

            \item To generate an inverted number triangle, sample output:
            \begin{verbatim}
1 2 3 4 5
1 2 3 4
1 2 3
1 2
1
            \end{verbatim}

            \framebreak

            \item To create a simple calculator using the \textttbg{switch} statement. First, define a character variable (call it \textttbg{op} for operator) using the \textttbg{char} keyword, then use \textttbg{scanf("\%c", \&op)}, the user shall input one of the following symbols: +, -, *, /. Then take two numbers (can be integers or floats) as user input. Finally, use the \textttbg{switch} keyword to perform addition, subtraction, multiplication or division based on the input to \textttbg{op}. If user inputs some unexpected character, then print \textttbg{invalid input}
        \end{enumerate}
    \end{frame}

\end{document}
