\documentclass[12pt, aspectratio=169]{beamer}

\input{../header}
\usepackage{minted}
\renewcommand{\theFancyVerbLine}{\ttfamily{\small\oldstylenums{\arabic{FancyVerbLine}}}}
\setminted{
    frame=none,
    bgcolor=gray!20,
    % linenos, 
    % numbers=left,
    % numbersep=5pt
    breaklines=true,
    fontsize=\fontsize{12pt}{13pt}\selectfont
}


\title{Structs in C}
\author{Md. Aminul Islam Shazid}
\date{26 Nov 2025}

\begin{document}

    {
		\setbeamertemplate{footline}{}    % NO FOOTLINE FOR THESE TWO FRAMES
		\addtocounter{framenumber}{-2}    % not counting the title page and the outline in frame numbers

		\begin{frame}
			\titlepage
		\end{frame}

		\begin{frame}{Outline}
            \vfill
            % \small
			\tableofcontents[subsectionstyle=hide]
            \vfill
		\end{frame}
	}

    
    \section{Introduction}



    \begin{frame}{Motivation}
    Many real-world things involve multiple pieces of information of different types.  
    Examples:
    \begin{itemize}
        \item A student has a name, ID, and CGPA
        \item A point has x and y coordinates
        \item A book has a title, author, and price
    \end{itemize}

    Arrays cannot store such heterogeneous data, so C provides structures to group diverse data elements.
    \end{frame}


    \begin{frame}{Introduction}
        \begin{itemize}
            \item A structure (also known as struct) is a user-defined composite data type
            \item It bundles several variables (called members) into a single logical unit
            \item A structure definition:
            \begin{itemize}
                \item Specifies the names and types of fields
                \item Describes a layout template
                \item Does not allocate memory by itself
                \item Memory is allocated only when structure variables are declared
            \end{itemize}
        \end{itemize}
    \end{frame}


    \begin{frame}[fragile]{Defining a Structure}
        Below a new type layout called \texttt{struct Student} is created:
        \begin{minted}{c}
struct Student {
    char name[50];
    int id;
    float cgpa;
};
        \end{minted}

        \begin{itemize}
            \item Structs are usually created outside and above the main function, so that all functions inside the \texttt{C} file can access the struct
            \item If a struct is created inside a function, then it behaves like a local variable and other functions can not use it
            \item One can also put a struct inside a header file for convenience
        \end{itemize}
    \end{frame}


    \begin{frame}[fragile]{Declaring Structure Variables}
        Three variable of the type \texttt{struct Student} have been defined below:
        \begin{minted}{c}
struct Student s;
struct Student t, class_monitor;
        \end{minted}

        \vspace{1em}
        NOTE: While structures are defined outside functions, structure variables are defined \emph{inside} functions.
    \end{frame}


        \begin{frame}[fragile]{Initializing Structs}
        \begin{minted}{c}
struct Student a = { "Alice", 1001, 3.90 };

struct Student b = {
    .name = "Bob",
    .id = 1002,
    .cgpa = 3.75
};
        \end{minted}

    \vspace{1em}
    Can also keep one or more fields uninitialzied:
    
        \begin{minted}{c}
struct Student c = {
    .id = 1003,
    .cgpa = 3.80
};
        \end{minted}
    \end{frame}


    \begin{frame}[fragile]{Accessing Members (Dot Operator)}
        \begin{minted}{c}
s.id = 101;
s.cgpa = 3.8;
        \end{minted}

    The dot operator accesses fields inside a structure variable.
    \end{frame}


    \begin{frame}{Example}
        \inputminted[linenos]{C}{../code-examples/10_01_struct-basic.c}
    \end{frame}


    \begin{frame}[fragile]{Arrays of Structs}
        \begin{minted}{c}
struct Student class[50];

class[0].name = "Alice";
class[0].id = 1001;
class[0].cgpa = 3.90;
        \end{minted}

        Useful for records, lists, and tables.
    \end{frame}


    \begin{frame}[fragile]{Nested Structures}
        A structure may contain another structure as a member:
        
        \begin{minted}{c}
struct Address {
    char city[30];
    int zip;
};

struct Person {
    char name[50];
    struct Address home;
};
        \end{minted}
    \end{frame}


    \section{Defining Types with \texttt{typedef}}

    \begin{frame}{The \texttt{typedef} Keyword}
        \begin{itemize}
            \item A structure type name often becomes long and repetitive
            \item Using \texttt{typedef} helps create cleaner and shorter type names
            \item Without typedef: \texttt{struct Student s;}
            \item With typedef: \texttt{Student s;}
        \end{itemize}

    This improves readability, especially in function prototypes and pointer-heavy code.
    \end{frame}


    \begin{frame}[fragile]{typedef Example}
    \begin{minted}{c}
typedef struct {
    int day;
    int month;
    int year;
} Date;

Date today;
today.day = 18;
    \end{minted}
    \end{frame}


    \begin{frame}{When typedef is Helpful}
        When it is useful:
        \begin{itemize}
            \item When the struct type is used frequently
            \item When building abstract data types
            \item When designing libraries or APIs
            \item When working with pointers to structs
        \end{itemize}
        \vspace{1em}
        When not mandatory:
        \begin{itemize}
            \item Very simple, rarely used struct types
            \item When emphasizing that a type is a struct (some coding styles prefer this)
        \end{itemize}
    \end{frame}


    \section{Pointers to Structs}


    \begin{frame}{Struct Pointers}
        A pointer to a struct is used when:
        \begin{itemize}
            \item Passing to functions efficiently
            \item Modifying the original struct
            \item Allocating structs dynamically
            \item Implementing dynamic data structures (linked lists, trees)
        \end{itemize}
    \end{frame}


    \begin{frame}[fragile]{Pointer to a Struct}
        \begin{minted}{c}
struct Student s, *p;

p = &s;
p->cgpa = 3.70;    // pointer access
        \end{minted}
        In the last line, we are setting the \texttt{cgpa} field to 3.70. This \texttt{cgpa} field belongs to the struct whose memory address is stored in \texttt{p}.
    \end{frame}


    \begin{frame}[fragile]{Dot vs Arrow}
        \begin{itemize}
            \item Dot: use with actual struct variables
            \item Arrow: use with struct pointers
        \end{itemize}

        \begin{minted}{c}
(*p).id = 20;
p->id = 20;   // preferred form when using with pointers
        \end{minted}
    \end{frame}


    \begin{frame}[fragile]{Passing Structs to Functions}
        Pass by value:
        \begin{minted}{c}
void show(struct Student s);
        \end{minted}

        \vspace{1em}
        Pass by pointer/reference (preferred for modification):
        \begin{minted}{c}
void update(struct Student *s) {
    s->cgpa += 0.1;
}
        \end{minted}
    \end{frame}


    \begin{frame}[fragile]{Returning a Struct}
        \begin{minted}{c}
typedef struct { int x, y; } Point;

Point make_point(int a, int b) {
    Point p = {a, b};
    return p;
}
        \end{minted}

        Useful for small structures like coordinates, dates, and configuration options.
    \end{frame}


    \begin{frame}[fragile]{Using const with Struct Pointers}
        \begin{minted}{c}
void print_student(const struct Student *s);
        \end{minted}

        \begin{itemize}
            \item Declares that the function will not modify the struct
            \item Good practice for safety and clear intent
        \end{itemize}
    \end{frame}


    % \begin{frame}[fragile]{Example: Basic Student Struct}
    % \end{frame}

    % \begin{frame}[fragile]{Example: Pointer to Struct}
    % \end{frame}

    % \begin{frame}[fragile]{Example: Array of Structs}
    % \end{frame}

    % \begin{frame}[fragile]{Example: Nested Struct Example}
    % \end{frame}

    % \begin{frame}[fragile]{Example: Passing Struct by Pointer}
    % \end{frame}

    % \begin{frame}[fragile]{Example: Returning a Struct}
    % \end{frame}

    % \begin{frame}[fragile]{Example: Using typedef with Struct}
    % \end{frame}


    \section{Exercise}

    \begin{frame}{Exercises}
        \begin{enumerate}
            \item Define a struct to hold the different parts (city, road name, house number etc.) of the address address of a person. Declare a variable of this struct type, set values to its elements by taking user input, then print the values
            \item Create a struct to hold some information on students including students' home address as defined in the last exercise. Declare a variable of this struct type, set values to its elements by taking user input, then print the values
            \item Create an array of the address struct. Using a loop, take user input into this array and then print the contents of the array
        \end{enumerate}
    \end{frame}

\end{document}
