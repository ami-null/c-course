\documentclass[12pt, aspectratio=169]{beamer}

\input{../header}
\usepackage{minted}
\renewcommand{\theFancyVerbLine}{\ttfamily{\small\oldstylenums{\arabic{FancyVerbLine}}}}
\setminted{
    frame=none,
    bgcolor=gray!20,
    % linenos, 
    % numbers=left,
    % numbersep=5pt
    breaklines=true,
    fontsize=\fontsize{12pt}{13pt}\selectfont
}


\title{File I/O in C}
\author{Md. Aminul Islam Shazid}
\date{06 Dec 2025}

\begin{document}

    {
		\setbeamertemplate{footline}{}    % NO FOOTLINE FOR THESE TWO FRAMES
		\addtocounter{framenumber}{-2}    % not counting the title page and the outline in frame numbers

		\begin{frame}
			\titlepage
		\end{frame}

		\begin{frame}{Outline}
            \vfill
            % \small
			\tableofcontents[subsectionstyle=hide]
            \vfill
		\end{frame}
	}

    \section{Introduction}


    \begin{frame}{File Input-Output (I/O)}
        Basic reasons we use files:
        \begin{itemize}
            \item Store results permanently
            \item Process large datasets that cannot be typed in manually
            \item Exchange data with other programs
        \end{itemize}

        Two broad file categories:
        \begin{itemize}
            \item Text files: store data as human-readable characters
            \item Binary files: store raw bytes (we will not cover details here)
        \end{itemize}
    \end{frame}


    \begin{frame}{FILE Pointer}
        \begin{itemize}
            \item Files are accessed using file pointers: \textttbg{FILE *}
            \item A \textttbg{FILE *} variable does not hold actual file content
            \item It holds information about the connection: current position, status, buffering, mode, etc.
        \end{itemize}
    \end{frame}


    \begin{frame}{Life Cycle of using a File}
        \begin{enumerate}
            \item Open the file using \textttbg{fopen()}
            \item Read from or write to the file
            \item Close the file using \textttbg{fclose()}
        \end{enumerate}

        Failing to close a file may lead to:
        \begin{itemize}
            \item Data not being fully written (due to buffering)
            \item Running out of available file descriptors
        \end{itemize}
    \end{frame}


    \begin{frame}[fragile]{Opening Files}
        To work with files, we must first open them using \textttbg{fopen()}.

        \begin{minted}{C}
FILE *fp = fopen("filename.txt", "mode");
        \end{minted}

        Basic modes:
        \begin{itemize}
            \item \textttbg{"r"}: open for reading; fails if file does not exist
            \item \textttbg{"w"}: open for writing; creates new file or overwrites existing file
            \item \textttbg{"a"}: open for appending (adding new contents to existing file); creates file if needed
        \end{itemize}
    \end{frame}


    \begin{frame}[fragile]{Opening Files (cont.)}
        Always check if opening succeeded:
        \begin{minted}{C}
if(fp == NULL){
    // error
}
        \end{minted}

        Common reasons for failure include:
        \begin{itemize}
            \item File does not exist (when using \textttbg{"r"})
            \item Insufficient permissions
            \item Incorrect path/file name
        \end{itemize}
    \end{frame}


    \begin{frame}{Closing Files}
        Closing files is done using \textttbg{fclose(fp);}.

        Reasons to close files:
        \begin{itemize}
            \item Ensures buffered output is actually written to disk
            \item Frees system resources
            \item Prevents file corruption
        \end{itemize}

        If a program ends without closing files, the OS usually closes them, but:
        \begin{itemize}
            \item This is unreliable
            \item Buffered content may be lost
            \item Considered bad programming practice
        \end{itemize}
    \end{frame}


    \section{Example}

    \begin{frame}[fragile]{Writing to a File}
        Writing must come first so the next example can read the file.

        Common functions:
        \begin{itemize}
            \item \textttbg{fprintf()}: write formatted text.
            \item \textttbg{fputs()}: write a string.
            \item \textttbg{fputc()}: write a single character.
        \end{itemize}
    \end{frame}


    \begin{frame}{Example: Writing to a File}
        \inputminted[linenos]{C}{../code-examples/11_01_writing-file.c}
    \end{frame}


    \begin{frame}[fragile]{Reading from a File}
        Now we read the file created by the previous program.

        Functions:
        \begin{itemize}
            \item \textttbg{fscanf()}: formatted reading
            \item \textttbg{fgets()}: read a line
            \item \textttbg{fgetc()}: read a single character
        \end{itemize}
    \end{frame}


    \begin{frame}{Example: Reading from a File}
        \inputminted[linenos]{C}{../code-examples/11_02_reading-file.c}
    \end{frame}


    \section{Summary}


    \begin{frame}{Basic Error Handling}
        Important situations to detect:
        \begin{itemize}
            \item \textbf{Failed \textttbg{fopen()}}: always check if pointer returned is \textttbg{NULL}
            \item \textbf{Reaching \textttbg{EOF}}: functions return special values (e.g., \textttbg{NULL} for \textttbg{fgets}, \textttbg{EOF} for \textttbg{fgetc})
            \item \textbf{Read errors}: typically uncommon in small programs but detectable using \textttbg{ferror()}
        \end{itemize}

        EOF vs Error:
        \begin{itemize}
            \item EOF means the file ended normally
            \item Errors indicate issues such as hardware failure or file being removed during reading
        \end{itemize}

        In basic programs, it is usually enough to check for \textttbg{NULL} or \textttbg{EOF}
    \end{frame}


    \begin{frame}{Summary}
        \begin{itemize}
            \item Use \textttbg{fopen()} to open files with the correct mode
            \item Perform reading or writing using \textttbg{fprintf()}, \textttbg{fputs()}, \textttbg{fgetc()}, \textttbg{fgets()}, or \textttbg{fscanf()}
            \item Always close files using \textttbg{fclose()}
            \item Always check if opening a file failed
        \end{itemize}

        These are the essential basics required before moving to advanced topics like binary files and random access.
    \end{frame}

\end{document}
