% \documentclass[12pt, handout]{beamer}
\documentclass[12pt, aspectratio=169]{beamer}

\usetheme{moloch}
\molochset{block=fill}

\usepackage{fontspec}
\setsansfont[
    UprightFont = Inter-Light,          % Use light as the normal font weight
    BoldFont = Inter-SemiBold,           % Use semibold for \textbf
    ItalicFont = Inter-LightItalic,      % Light italic for \textit
    BoldItalicFont = Inter-SemiBoldItalic % Semibold italic for \textbf with \textit
]{Inter}[RawFeature={+ss04, +ss03, +dlig, +tnum}]
% Inter font stylistic sets:
% ss01: alternate digits for 3, 4, 6, 8
% ss02: for disambiguation (with zero) in places like "Ill" or "O0" etc
% ss04: for disambiguation (without zero) in places like "Ill" etc
% ss03: round comma, quation marks

\usepackage{upquote}
\usepackage{microtype}
\UseMicrotypeSet[protrusion]{basicmath}    % disable protrusion for tt fonts

\usepackage{amsmath}
\usepackage{amssymb}
\usepackage{unicode-math}
\setmathfont{Erewhon Math}[Scale=1.14]

\usepackage{hyperref}
\pdfstringdefDisableCommands{\def\translate#1{#1}}
\usepackage{bookmark}
\usepackage{url}

\usepackage{natbib}
\usepackage{appendixnumberbeamer}
\usepackage{enumerate}
% \usepackage{enumitem}    % it is giving the error: TeX capacity exceeded
% \usepackage{footnotehyper}
\usepackage{graphicx}
\usepackage{caption}
% \usepackage{subcaption}
\usepackage{booktabs}
\usepackage{makecell}
\usepackage{array}
\newcolumntype{H}{>{\setbox0=\hbox\bgroup\let\pm\relax}c<{\egroup}@{}}
% \newcolumntype{H}{>{\setbox0=\hbox\bgroup}c<{\egroup}}% <--- removed @{}
% https://tex.stackexchange.com/questions/567724/can-i-hide-a-table-column-with-the-s-type-from-siunitx
% https://tex.stackexchange.com/questions/414143/hide-column-without-adding-whitespace-to-table


\definecolor{airforceblue}{rgb}{0.36, 0.54, 0.66}
\hypersetup{
    colorlinks=true,
    linkcolor={mDarkTeal},    % this colour is defined by the moloch theme
    filecolor={Maroon},
    citecolor={airforceblue!120},
    urlcolor={airforceblue!140},
    pdfcreator={xelatex},
    bookmarksopen=true,    % Expand bookmarks in the PDF
    bookmarksnumbered=true % Include numbering in bookmarks
}

\bibliographystyle{apalike}

\let\oldcite=\cite
\renewcommand{\cite}[1]{\textcolor{airforceblue!120}{\oldcite{#1}}}
\let\oldcitet=\citet
\renewcommand{\citet}[1]{\textcolor{airforceblue!120}{\oldcitet{#1}}}
\let\oldcitep=\citep
\renewcommand{\citep}[1]{\textcolor{airforceblue!120}{\oldcitep{#1}}}


% \setlength{\leftmargini}{0em}
\setbeamercolor{page number in head/foot}{fg=gray}
\setbeamertemplate{footline}[frame number]
\setbeamertemplate{itemize items}[circle]
\setbeamertemplate{enumerate items}[circle]
\setbeamertemplate{sections/subsections in toc}[circle]
\setbeamertemplate{frametitle continuation}[from second][(cont.)]
\setbeamercovered{transparent}
\beamertemplatenavigationsymbolsempty


\AtBeginSubsection[]{
    {
        \begin{frame}[noframenumbering, plain]
            \subsectionpage
        \end{frame}
    }
}


\newcommand\Wider[2][4em]{%
    \makebox[\linewidth][c]{%
        \begin{minipage}{\dimexpr\textwidth+#1\relax}
            % \raggedright#2
            \centering#2
        \end{minipage}%
    }%
}

% \newenvironment{myitemize}{
%     \begin{itemize}
%         \vspace{1em}
%         \setlength{\itemsep}{0.7\baselineskip}
% }{
%         \vspace{1em}
%     \end{itemize}
% }

% \newenvironment{myenumerate}{
%     \begin{enumerate}
%         \vspace{1em}
%         \setlength{\itemsep}{0.7\baselineskip}
% }{
%         \vspace{1em}
%     \end{enumerate}
% }


\title{Introduction to the C Language}
\author{Md. Aminul Islam Shazid}
\date{}


\begin{document}
    {
		\setbeamertemplate{footline}{}    % NO FOOTLINE FOR THESE TWO FRAMES
		\addtocounter{framenumber}{-2}    % not counting the title page and the outline in frame numbers

		\begin{frame}
			\titlepage
		\end{frame}

		\begin{frame}{Outline}
			\tableofcontents[subsectionstyle=hide]
		\end{frame}
	}

    \section{History}

    \begin{frame}{Creation of C}
        \begin{itemize}
            \item Created by Dennis Ritchie in close collaboration with Ken Thompson at the Bell Labs in the early 1970s
            \item The first operating system written in C is Unix which ran on the 16bit \href{https://en.wikipedia.org/wiki/PDP-11}{PDP-11} computer
            
        \end{itemize}
    \end{frame}
    
    \begin{frame}{Evolution of C}
        \begin{itemize}
            \item In 1978, Dennis Ritchie and Brian Kernighan released the book titled ``The C Programming Language"
            \item C was standardized by ANSI (known as C89) in 1989, this was adopted by ISO later in 1990 (known as C90)
            \item The latest version of the standard is C23
            \item Since 2000, consistently ranked top four in the \href{https://en.wikipedia.org/wiki/TIOBE_index}{TIOBE index}
        \end{itemize}
    \end{frame}

	\section{Features of C}
	
	\begin{frame}{Basic Features}
		\begin{itemize}
            \item High level (compared to assembly or machine code) 
            \item Also provides low level access
                \begin{itemize}
                    \item Allows writing inline assembly code
                    \item Provides direct access to memory management
                \end{itemize}
            \item Fewer keywords compared to other contemporary languages
        \end{itemize}
	\end{frame}

    \begin{frame}{Basic Features (cont.)}
        \begin{itemize}
            \item Compiles to native machine code -- runs very fast
            \item Statically typed, supports user defined data types
            \item Compund data types -- structs and unions
            \item Standard library as well as third party libraries
        \end{itemize}
    \end{frame}

    \begin{frame}{Shortcomings}
        \begin{itemize}
            \item Not object oriented
            \item Lacks exceptions
            \item No garbage collection
            \item Can be unsafe if not careful
            \begin{itemize}
                \item No range-checking
                \item Limited type checking at compile time
                \item No type checking at runtime
            \end{itemize}
        \end{itemize}
    \end{frame}

    \begin{frame}{Use Cases}
        \begin{itemize}
            \item Operating systems, such as Linux
            \item Many software
            \begin{itemize}
                \item Database: PostgreSQL, SQLite, MySQL
                \item VLC Player
                \item Apache web server, Nginx
            \end{itemize}
            \item Embedded systems (microcontrollers)
            \item Other higher level programming languages
            \begin{itemize}
                \item CPython
                \item Ruby
                \item PHP
                \item JVM (Java, Kotlin etc.)
            \end{itemize}
        \end{itemize}
    \end{frame}


    \section{Getting Started with C}

    \begin{frame}{Ingredients}
        \begin{itemize}
            \item C source file (plain text file with .c extension)
            \item Compiler (ideally, also a debugger)
            \item Text editor
            \item Terminal (aka Command Prompt on Windows)
        \end{itemize}
    \end{frame}

    \begin{frame}{Compilers and Debuggers}
        \begin{itemize}
            \item Compiler turns the C code into machine code
            \item Debugger helps in debugging a compiled program
            \item Most popular C compilers include
            \begin{itemize}
                \item GNU C Compiler and GNU Debuger (GCC and GDB)
                \item Clang
                \item Microsoft Visual C Compiler (MSVC)
            \end{itemize}
        \end{itemize}
    \end{frame}

    \begin{frame}{Ingredients (cont.)}
        To make life easier, use an IDE (integrated development environment), it combines:
        \begin{itemize}
            \item Source editing
            \item Invoking compiler
            \item Debugging
            \item Version control system (VCM), for example git or subversion
        \end{itemize}

        Most modern IDEs also provide ``intellisense" as well as perform static analysis on source codes.
    \end{frame}

    \begin{frame}{Environrment Setup on Windows}
        \begin{itemize}
            \item Easiest option: download \href{https://www.codeblocks.org/downloads/binaries/}{CodeBlocks} with MinGW which includes GCC and GDB
            \item Microsoft Visual Studio with MSVC (much bigger download size)
        \end{itemize}
    \end{frame}

    \begin{frame}{Installing Only GCC on Windows}
        In order to use an IDE that does not bundle a compiler, one needs to install the compiler separately. Three options:
        \begin{itemize}
            \item Download \href{https://www.mingw-w64.org/}{MinGW-W64} using \href{https://www.msys2.org/}{MSYS2}
            \item Download MinGW-W64 standalone from \href{https://winlibs.com/}{WinLibs}
            \item Installing via \href{https://www.cygwin.com/}{Cygwin}
        \end{itemize}
    \end{frame}

    \begin{frame}{Environment Setup on Linux}
        \begin{itemize}
            \item Install the \texttt{gcc} package using the package manager
            \item Then install the \texttt{codeblocks} package or any other suitable IDE
        \end{itemize}
    \end{frame}

    \begin{frame}{Environment Setup on MacOS}
        \begin{itemize}
            \item Install XCode, comes with the Clang compiler
            \item Can be installed from the App Store
            \item If only the compilers are wanted, then run the follwing in the terminal: \texttt{xcode-select~--install}
        \end{itemize}
    \end{frame}

    \begin{frame}{Crossplatform IDEs and Code Editors}
        \begin{itemize}
            \item Codeblocks
            \item Visual Studio Code (requires seperate extension for C/C++)
            \item Clion (free for non-commercial use only)
            \item Eclipse IDE
        \end{itemize}
    \end{frame}

    \section{Our First C Program}

    \begin{frame}{Hello World in C}
        \inputminted[
            frame=none,
            bgcolor=gray!6,
            linenos, 
            breaklines=true,
            fontsize=\fontsize{12pt}{13pt}\selectfont,
            % firstline=37,
            % lastline=54,
            % firstnumber=1
        ]{c}{../code-examples/01_hello-world.c}
    \end{frame}

    \section*{Thank you.}

    % \appendix

    % \begingroup
    % \renewcommand{\section}[2]{}%
    % \begin{frame}[allowframebreaks]{References}
    %     \def\bibfont{\footnotesize}
    %     %\Wider{
    %         \bibliography{refdb.bib}
    %     %}
    % \end{frame}
    % \endgroup

    % \begin{frame}{Appendix}
    %     test appendix frame
    % \end{frame}

\end{document}
