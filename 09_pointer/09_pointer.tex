\documentclass[12pt, aspectratio=169]{beamer}

\input{../header}
\usepackage{minted}
\renewcommand{\theFancyVerbLine}{\ttfamily{\small\oldstylenums{\arabic{FancyVerbLine}}}}
\setminted{
    frame=none,
    bgcolor=gray!20,
    % linenos, 
    % numbers=left,
    % numbersep=5pt
    breaklines=true,
    fontsize=\fontsize{12pt}{13pt}\selectfont
}


\title{Pointers in C}
\author{Md. Aminul Islam Shazid}
\date{24 Nov 2025}

\begin{document}

    {
		\setbeamertemplate{footline}{}    % NO FOOTLINE FOR THESE TWO FRAMES
		\addtocounter{framenumber}{-2}    % not counting the title page and the outline in frame numbers

		\begin{frame}
			\titlepage
		\end{frame}

		\begin{frame}{Outline}
            \vfill
            \small
			\tableofcontents[subsectionstyle=hide]
            \vfill
		\end{frame}
	}

    
    \section{Introduction}


    \begin{frame}[fragile]{What is a Pointer?}
        \begin{itemize}
            \item Every variable has an address in memory
            \item A pointer stores the memory address of another variable
            \item A pointer itself is a variable and has its own memory address
        \end{itemize}
    \end{frame}


    \begin{frame}[fragile]{Defining a Pointer}
        Pointers are defined by putting a \texttt{*} before the pointer's name:
        \begin{minted}{c}
int x = 10;
int *p = &x;    // p stores the address of x
        \end{minted}
    \end{frame}


    \begin{frame}[fragile]{Different Types of Pointers}
        \begin{minted}{c}
int *p;     // pointer to int
char *c;    // pointer to char
float *f;   // pointer to float
        \end{minted}
    \end{frame}


    \begin{frame}[fragile]{Dereferencing a Pointer}
        Dereferencing means accessing the value stored at an address:
    \begin{minted}{c}
int x = 10;
int *p = &x;

printf("%d", *p);  // prints 10
        \end{minted}
    \end{frame}


    \begin{frame}{Example: Pointer Basics}
        \inputminted[linenos]{C}{../code-examples/09_01_pointer-basics.c}
    \end{frame}


    \section{Pointer Arithmetic}


    \begin{frame}[fragile]{Pointer Arithmetic}
        Adding or subtracting integers moves a pointer by multiples of the pointed type's size.

        \inputminted[linenos]{C}{../code-examples/09_02_pointer-arithmetic-basic.c}
    \end{frame}


    \begin{frame}[fragile]{Array Name as Pointer}
        The name of an array holds the memory address to the first element of that array:
        \inputminted[linenos]{C}{../code-examples/09_03_arrays-and-pointers.c}
    \end{frame}


    \begin{frame}[fragile]{Pointer Difference}
        \begin{minted}{c}
int a[5];
int *p = &a[0];
int *q = &a[3];

int diff = q - p;   // diff = 3
        \end{minted}
    \end{frame}


\section{Pointers with Function}


    \begin{frame}[fragile]{Example: Pointer as Function Parameter}
        \inputminted[linenos]{C}{../code-examples/09_04_pointer-function-increment.c}
    \end{frame}


    \begin{frame}[fragile]{Example: Swap Using Pointers}
        \inputminted[linenos]{C}{../code-examples/09_05_pointer-function-swap.c}
    \end{frame}


    \begin{frame}[fragile]{Example: Array Passed to Function}
        \inputminted[linenos]{C}{../code-examples/09_06_passing-array-to-function.c}
    \end{frame}


    \begin{frame}{Example: Modify Array in Function}
        \inputminted[linenos]{C}{../code-examples/09_07_function-modify-array.c}
    \end{frame}


    % \section{Exercise}

    % \begin{frame}{Exercises}
    %     \begin{enumerate}
    %         \item 
    %     \end{enumerate}
    % \end{frame}

\end{document}
