\documentclass[12pt, aspectratio=169]{beamer}

\input{../header}
\usepackage{minted}
\renewcommand{\theFancyVerbLine}{\ttfamily{\small\oldstylenums{\arabic{FancyVerbLine}}}}
\setminted{
    frame=none,
    bgcolor=gray!20,
    % linenos, 
    % numbers=left,
    % numbersep=5pt
    breaklines=true,
    fontsize=\fontsize{12pt}{13pt}\selectfont
}


\title{Functions in C}
\author{Md. Aminul Islam Shazid}
\date{22 Nov 2025}

\begin{document}

    {
		\setbeamertemplate{footline}{}    % NO FOOTLINE FOR THESE TWO FRAMES
		\addtocounter{framenumber}{-2}    % not counting the title page and the outline in frame numbers

		\begin{frame}
			\titlepage
		\end{frame}

		\begin{frame}{Outline}
            \vfill
            \small
			\tableofcontents[subsectionstyle=hide]
            \vfill
		\end{frame}
	}

    \section{Introduction}

    \begin{frame}{Introduction to Functions in C}
        \begin{itemize}
            \item A function is a \emph{reusable} block of code that performs a specific task
            \item Functions help organize programs into smaller and manageable sections
            \item The \texttt{main()} function is the entry point to every C program
        \end{itemize}
    \end{frame}


    \begin{frame}{Why Use Functions}
        \begin{itemize}
            \item To avoid repeating the same code
            \item To make programs easier to understand and maintain
            \item To divide a large problem into smaller parts
            \item To allow reusability of code
        \end{itemize}
    \end{frame}


    \begin{frame}{Advantages of Using Functions}
        \begin{itemize}
            \item Reduces code duplication
            \item Enhances readability
            \item Helps debugging and testing individual parts easily
            \item Supports modular program design
        \end{itemize}
    \end{frame}


    \section{Syntax}


    \begin{frame}[fragile]{Syntax of a Function}
        \begin{minted}{c}
return_type function_name(parameter_list) {
    // body of the function
    return the_return_value;  // optional
}
        \end{minted}

        \begin{itemize}
            \item Function declaration tells the compiler about the function
            \item Function definition contains the actual code
            \item Function call transfers control to the function
        \end{itemize}
    \end{frame}


    \begin{frame}[fragile]{Example: Function with No Parameters}
        \begin{minted}{c}
void greet() {
    printf("Hello, World!");
}
        \end{minted}
    \end{frame}


    \begin{frame}[fragile]{Example: Function with One Parameter}
        \begin{minted}{c}
void printNumber(int n) {
    printf("The number is %d", n);
}
        \end{minted}
    \end{frame}


    \begin{frame}[fragile]{Return Type and Return Value}
        \begin{minted}{c}
int square(int n) {
    return n * n;
}
        \end{minted}


        \begin{itemize}
            \item The return type defines the type of value a function returns
            \item The return statement sends a value back to the calling code
        \end{itemize}
    \end{frame}


    \begin{frame}[fragile]{Example: Function with Multiple Parameters}
        \begin{minted}{c}
int add(int a, int b) {
    return a + b;
}
        \end{minted}
    \end{frame}


    \begin{frame}[fragile]{Calling Functions}
        \begin{minted}{c}
greet();             // no parameter
printNumber(5);      // one parameter
sum = add(4, 6);     // multiple parameters
        \end{minted}
    \end{frame}



    \begin{frame}{Types of Functions}
        \begin{itemize}
            \item Library functions - predefined in header files like \texttt{printf()}, \texttt{scanf()}, \texttt{sqrt()}
            \item User-defined functions - created by the programmer
        \end{itemize}
    \end{frame}


    \section{Examples}


    \begin{frame}{Example: Function with No Parameters and No Return Value}
        \inputminted[
            linenos, 
        ]{c}{../code-examples/08_01_greet.c}
        When a function returns no value, its return type is \texttt{void}.
    \end{frame}


    \begin{frame}{Example: Function with One Parameter and No Return Value}
        \inputminted[
            linenos, 
        ]{c}{../code-examples/08_02_print-square.c}
    \end{frame}


    \begin{frame}{Example: Function with One Parameter and a Return Value}
        \inputminted[
            linenos, 
        ]{c}{../code-examples/08_03_print-square-return.c}
        Since this function returns an integer value, its return type is \texttt{int}.
    \end{frame}


    \begin{frame}{Example: Function with Multiple Parameters and Return Value}
        \inputminted[
            linenos, 
        ]{c}{../code-examples/08_04_add-two-int.c}
    \end{frame}


    \begin{frame}{Example: Function Returning a Value Without Parameters}
        \inputminted[
            linenos, 
        ]{c}{../code-examples/08_05_meaning-of-life.c}
    \end{frame}


    \begin{frame}{Example: Function That Calls Another Function}
        \inputminted[
            linenos, 
        ]{c}{../code-examples/08_06_calls-another.c}
    \end{frame}


    \begin{frame}[fragile]{Recursive Functions}
        \begin{itemize}
            \item A recursive function calls itself
            \item Must have a base case to stop recursion
        \end{itemize}

        \begin{minted}[linenos]{c}
int factorial(int n){
    if(n == 0){
        return 1;
    } else{
        return n*factorial(n-1);
    }
}
        \end{minted}
    \end{frame}


    \section{Function Prototype}

    \begin{frame}[fragile]{What Is a Function Prototype?}
        A function prototype is a declaration of a function without its body.

        \begin{minted}{c}
int add(int a, int b);     // prototype only
        \end{minted}

        \begin{itemize}
            \item Ends with a semicolon.
            \item Parameter names are optional:
            \begin{minted}{c}
int add(int, int);
            \end{minted}
            \item Must match the definition exactly: return type, name, and parameter types.
        \end{itemize}
    \end{frame}


    \begin{frame}{Why Function Prototypes?}
        \begin{itemize}
            \item The compiler reads code top to bottom
            \item A function must be known before it is called
            \item If \texttt{main()} calls a function defined later, the compiler has no information about that function
            \item A prototype supplies the missing information: return type, function name, and parameter types
            \item Prototypes enable type checking at compile time
        \end{itemize}
    \end{frame}


    \begin{frame}{Compiler Perspective}
        The compiler must know:
        \begin{itemize}
            \item How many arguments a function takes
            \item What their types are
            \item What return type to expect
        \end{itemize}
    \end{frame}


    \begin{frame}{Example: Prototypes in a Single C File}
        \inputminted[
            linenos, 
        ]{c}{../code-examples/08_07_function-prototype_single-file.c}
    \end{frame}


    \section{Header Files}
    
    \begin{frame}{Introduction}
        \begin{itemize}
            \item Sometimes it may be necessary to use the same functions/constants across multiple C files in a project
            \item In such cases, it is infeasible to define the functions/constants in each C file
            \item Header files (files with \texttt{.h} extension) help reuse function/constant definition across multiple C files
            \item Header files have no \texttt{main()} function
        \end{itemize}
    \end{frame}

    
    \begin{frame}{Example Header File}
        \inputminted[
            linenos, 
        ]{c}{../code-examples/utils.h}
        Save this file as \texttt{utils.h} (you can use any name you want).
    \end{frame}


    \begin{frame}{Using User-Created Header File}
        \inputminted[
            linenos, 
        ]{c}{../code-examples/08_08_using-header-file.c}
    \end{frame}


    \section{Exercise}

    \begin{frame}{Exercises}
        \begin{enumerate}
            \item Write a function to find the maximum of two numbers
            \item Write a function that checks if an integer is even or odd
            \item Write a function that takes three numbers and returns their average
            \item Write a recursive function to calculate the sum of digits of an integer
            \item Write a recursive function to calculate the GCD of two integers
            \item Write a function that checks whether a given integer is prime
            \item Write a function to print all prime numbers between 1 and 100
        \end{enumerate}
    \end{frame}

\end{document}
