\documentclass[12pt, aspectratio=169]{beamer}

\usetheme{moloch}
\molochset{block=fill}

\usepackage{fontspec}
\setsansfont[
    UprightFont = Inter-Light,          % Use light as the normal font weight
    BoldFont = Inter-SemiBold,           % Use semibold for \textbf
    ItalicFont = Inter-LightItalic,      % Light italic for \textit
    BoldItalicFont = Inter-SemiBoldItalic % Semibold italic for \textbf with \textit
]{Inter}[RawFeature={+ss04, +ss03, +dlig, +tnum}]
% Inter font stylistic sets:
% ss01: alternate digits for 3, 4, 6, 8
% ss02: for disambiguation (with zero) in places like "Ill" or "O0" etc
% ss04: for disambiguation (without zero) in places like "Ill" etc
% ss03: round comma, quation marks

\usepackage{upquote}
\usepackage{microtype}
\UseMicrotypeSet[protrusion]{basicmath}    % disable protrusion for tt fonts

\usepackage{amsmath}
\usepackage{amssymb}
\usepackage{unicode-math}
\setmathfont{Erewhon Math}[Scale=1.14]

\usepackage{hyperref}
\pdfstringdefDisableCommands{\def\translate#1{#1}}
\usepackage{bookmark}
\usepackage{url}

\usepackage{natbib}
\usepackage{appendixnumberbeamer}
\usepackage{enumerate}
% \usepackage{enumitem}    % it is giving the error: TeX capacity exceeded
% \usepackage{footnotehyper}
\usepackage{graphicx}
\usepackage{caption}
% \usepackage{subcaption}
\usepackage{booktabs}
\usepackage{makecell}
\usepackage{array}
\newcolumntype{H}{>{\setbox0=\hbox\bgroup\let\pm\relax}c<{\egroup}@{}}
% \newcolumntype{H}{>{\setbox0=\hbox\bgroup}c<{\egroup}}% <--- removed @{}
% https://tex.stackexchange.com/questions/567724/can-i-hide-a-table-column-with-the-s-type-from-siunitx
% https://tex.stackexchange.com/questions/414143/hide-column-without-adding-whitespace-to-table


\definecolor{airforceblue}{rgb}{0.36, 0.54, 0.66}
\hypersetup{
    colorlinks=true,
    linkcolor={mDarkTeal},    % this colour is defined by the moloch theme
    filecolor={Maroon},
    citecolor={airforceblue!120},
    urlcolor={airforceblue!140},
    pdfcreator={xelatex},
    bookmarksopen=true,    % Expand bookmarks in the PDF
    bookmarksnumbered=true % Include numbering in bookmarks
}

\bibliographystyle{apalike}

\let\oldcite=\cite
\renewcommand{\cite}[1]{\textcolor{airforceblue!120}{\oldcite{#1}}}
\let\oldcitet=\citet
\renewcommand{\citet}[1]{\textcolor{airforceblue!120}{\oldcitet{#1}}}
\let\oldcitep=\citep
\renewcommand{\citep}[1]{\textcolor{airforceblue!120}{\oldcitep{#1}}}


% \setlength{\leftmargini}{0em}
\setbeamercolor{page number in head/foot}{fg=gray}
\setbeamertemplate{footline}[frame number]
\setbeamertemplate{itemize items}[circle]
\setbeamertemplate{enumerate items}[circle]
\setbeamertemplate{sections/subsections in toc}[circle]
\setbeamertemplate{frametitle continuation}[from second][(cont.)]
\setbeamercovered{transparent}
\beamertemplatenavigationsymbolsempty


\AtBeginSubsection[]{
    {
        \begin{frame}[noframenumbering, plain]
            \subsectionpage
        \end{frame}
    }
}


\newcommand\Wider[2][4em]{%
    \makebox[\linewidth][c]{%
        \begin{minipage}{\dimexpr\textwidth+#1\relax}
            % \raggedright#2
            \centering#2
        \end{minipage}%
    }%
}

% \newenvironment{myitemize}{
%     \begin{itemize}
%         \vspace{1em}
%         \setlength{\itemsep}{0.7\baselineskip}
% }{
%         \vspace{1em}
%     \end{itemize}
% }

% \newenvironment{myenumerate}{
%     \begin{enumerate}
%         \vspace{1em}
%         \setlength{\itemsep}{0.7\baselineskip}
% }{
%         \vspace{1em}
%     \end{enumerate}
% }
\usepackage{minted}
\renewcommand{\theFancyVerbLine}{\ttfamily{\small\oldstylenums{\arabic{FancyVerbLine}}}}
\setminted{
    frame=none,
    bgcolor=gray!20,
    % linenos, 
    breaklines=true,
    fontsize=\fontsize{12pt}{13pt}\selectfont
}


\title{Functions in C}
\author{}
\date{}

\begin{document}

    {
		\setbeamertemplate{footline}{}    % NO FOOTLINE FOR THESE TWO FRAMES
		\addtocounter{framenumber}{-2}    % not counting the title page and the outline in frame numbers

		\begin{frame}
			\titlepage
		\end{frame}

		\begin{frame}{Outline}
            % \vfill
            \small
			\tableofcontents[subsectionstyle=hide]
            % \vfill
		\end{frame}
	}

    \begin{frame}{Introduction to Functions in C}
    \begin{itemize}
        \item A function is a \emph{reusable} block of code that performs a specific task
        \item Functions help organize programs into smaller and manageable sections
        \item The \texttt{main} function is the entry point of every C program
    \end{itemize}
    \end{frame}


    \begin{frame}{Why Use Functions}
    \begin{itemize}
        \item To avoid repeating the same code
        \item To make programs easier to understand and maintain
        \item To divide a large problem into smaller parts
        \item To allow reusability of code
    \end{itemize}
    \end{frame}


    \begin{frame}{Advantages of Using Functions}
    \begin{itemize}
        \item Reduces code duplication
        \item Enhances readability
        \item Helps debugging and testing individual parts easily
        \item Supports modular program design
    \end{itemize}
    \end{frame}


    \begin{frame}[fragile]{Syntax of a Function}
    \begin{minted}{c}
return_type function_name(parameter_list) {
    // body of the function
    return the_return_value;  // optional
}
    \end{minted}

    \begin{itemize}
        \item Function declaration tells the compiler about the function
        \item Function definition contains the actual code
        \item Function call transfers control to the function
    \end{itemize}
    \end{frame}


    \begin{frame}[fragile]{Example: Function with No Parameters}
        \begin{minted}{c}
void greet() {
    printf("Hello, World!");
}
        \end{minted}
    \end{frame}


    \begin{frame}[fragile]{Example: Function with One Parameter}
    \begin{minted}{c}
void printNumber(int n) {
    printf("The number is %d", n);
}
\end{minted}
    \end{frame}


    \begin{frame}[fragile]{Example: Function with Multiple Parameters}
        \begin{minted}{c}
int add(int a, int b) {
    return a + b;
}
        \end{minted}
    \end{frame}


    \begin{frame}[fragile]{Calling Functions}
        \begin{minted}{c}
greet();             // no parameter
printNumber(5);      // one parameter
sum = add(4, 6);     // multiple parameters
        \end{minted}
    \end{frame}


    \begin{frame}[fragile]{Return Type and Return Value}
        \begin{minted}{c}
int square(int n) {
    return n * n;
}
        \end{minted}


    \begin{itemize}
        \item The return type defines the type of value a function returns
        \item The return statement sends a value back to the calling code
    \end{itemize}
    \end{frame}


    \begin{frame}{Types of Functions}
        \begin{itemize}
            \item Library functions - predefined in header files like ``printf()'', ``scanf()'', ``sqrt()''
            \item User-defined functions - created by the programmer
        \end{itemize}
    \end{frame}


    \begin{frame}[fragile]{Recursive Functions}
        \begin{itemize}
            \item A recursive function calls itself
            \item Must have a base case to stop recursion
        \end{itemize}

        \begin{minted}[linenos]{c}
int factorial(int n){
    if(n == 0){
        return 1;
    } else{
        return n * factorial(n - 1);
    }
}
        \end{minted}
    \end{frame}


    \begin{frame}{Example: Function with No Parameters and No Return Value}
        \inputminted[
            linenos, 
        ]{c}{../code-examples/08_01_greet.c}
    \end{frame}


    \begin{frame}{Example: Function with One Parameter and No Return Value}
        \inputminted[
            linenos, 
        ]{c}{../code-examples/08_02_print-square.c}
    \end{frame}


    \begin{frame}{Example: Function with One Parameter and a Return Value}
        \inputminted[
            linenos, 
        ]{c}{../code-examples/08_03_print-square-return.c}
    \end{frame}


    \begin{frame}{Example: Function with Multiple Parameters and Return Value}
        \inputminted[
            linenos, 
        ]{c}{../code-examples/08_04_add-two-int.c}
    \end{frame}


    \begin{frame}{Exercises}
        \begin{itemize}
            \item Write a function to find the maximum of two numbers
            \item Write a function that checks if an integer is even or odd
            \item Write a function that takes three numbers and returns their average
            \item Write a recursive function to calculate the sum of digits of an integer
            \item Write a recursive function to calculate the GCD of two integers
            \item Write a function that checks whether a given integer is prime
            \item Write a function to print all prime numbers between 1 and 100
        \end{itemize}
    \end{frame}

\end{document}
