% \documentclass[12pt, handout]{beamer}
\documentclass[12pt, aspectratio=169]{beamer}

\usetheme{moloch}
\molochset{block=fill}

\usepackage{fontspec}
\setsansfont[
    UprightFont = Inter-Light,          % Use light as the normal font weight
    BoldFont = Inter-SemiBold,           % Use semibold for \textbf
    ItalicFont = Inter-LightItalic,      % Light italic for \textit
    BoldItalicFont = Inter-SemiBoldItalic % Semibold italic for \textbf with \textit
]{Inter}[RawFeature={+ss04, +ss03, +dlig, +tnum}]
% Inter font stylistic sets:
% ss01: alternate digits for 3, 4, 6, 8
% ss02: for disambiguation (with zero) in places like "Ill" or "O0" etc
% ss04: for disambiguation (without zero) in places like "Ill" etc
% ss03: round comma, quation marks

\usepackage{upquote}
\usepackage{microtype}
\UseMicrotypeSet[protrusion]{basicmath}    % disable protrusion for tt fonts

\usepackage{amsmath}
\usepackage{amssymb}
\usepackage{unicode-math}
\setmathfont{Erewhon Math}[Scale=1.14]

\usepackage{hyperref}
\pdfstringdefDisableCommands{\def\translate#1{#1}}
\usepackage{bookmark}
\usepackage{url}

\usepackage{natbib}
\usepackage{appendixnumberbeamer}
\usepackage{enumerate}
% \usepackage{enumitem}    % it is giving the error: TeX capacity exceeded
% \usepackage{footnotehyper}
\usepackage{graphicx}
\usepackage{caption}
% \usepackage{subcaption}
\usepackage{booktabs}
\usepackage{makecell}
\usepackage{array}
\newcolumntype{H}{>{\setbox0=\hbox\bgroup\let\pm\relax}c<{\egroup}@{}}
% \newcolumntype{H}{>{\setbox0=\hbox\bgroup}c<{\egroup}}% <--- removed @{}
% https://tex.stackexchange.com/questions/567724/can-i-hide-a-table-column-with-the-s-type-from-siunitx
% https://tex.stackexchange.com/questions/414143/hide-column-without-adding-whitespace-to-table


\definecolor{airforceblue}{rgb}{0.36, 0.54, 0.66}
\hypersetup{
    colorlinks=true,
    linkcolor={mDarkTeal},    % this colour is defined by the moloch theme
    filecolor={Maroon},
    citecolor={airforceblue!120},
    urlcolor={airforceblue!140},
    pdfcreator={xelatex},
    bookmarksopen=true,    % Expand bookmarks in the PDF
    bookmarksnumbered=true % Include numbering in bookmarks
}

\bibliographystyle{apalike}

\let\oldcite=\cite
\renewcommand{\cite}[1]{\textcolor{airforceblue!120}{\oldcite{#1}}}
\let\oldcitet=\citet
\renewcommand{\citet}[1]{\textcolor{airforceblue!120}{\oldcitet{#1}}}
\let\oldcitep=\citep
\renewcommand{\citep}[1]{\textcolor{airforceblue!120}{\oldcitep{#1}}}


% \setlength{\leftmargini}{0em}
\setbeamercolor{page number in head/foot}{fg=gray}
\setbeamertemplate{footline}[frame number]
\setbeamertemplate{itemize items}[circle]
\setbeamertemplate{enumerate items}[circle]
\setbeamertemplate{sections/subsections in toc}[circle]
\setbeamertemplate{frametitle continuation}[from second][(cont.)]
\setbeamercovered{transparent}
\beamertemplatenavigationsymbolsempty


\AtBeginSubsection[]{
    {
        \begin{frame}[noframenumbering, plain]
            \subsectionpage
        \end{frame}
    }
}


\newcommand\Wider[2][4em]{%
    \makebox[\linewidth][c]{%
        \begin{minipage}{\dimexpr\textwidth+#1\relax}
            % \raggedright#2
            \centering#2
        \end{minipage}%
    }%
}

% \newenvironment{myitemize}{
%     \begin{itemize}
%         \vspace{1em}
%         \setlength{\itemsep}{0.7\baselineskip}
% }{
%         \vspace{1em}
%     \end{itemize}
% }

% \newenvironment{myenumerate}{
%     \begin{enumerate}
%         \vspace{1em}
%         \setlength{\itemsep}{0.7\baselineskip}
% }{
%         \vspace{1em}
%     \end{enumerate}
% }


\title{Introduction to Programming}
\author{Md. Aminul Islam Shazid}
\date{}


\begin{document}
    {
		\setbeamertemplate{footline}{}    % NO FOOTLINE FOR THESE TWO FRAMES
		\addtocounter{framenumber}{-2}    % not counting the title page and the outline in frame numbers

		\begin{frame}
			\titlepage
		\end{frame}

		\begin{frame}{Outline}
            \vfill
			\tableofcontents[subsectionstyle=hide]
            \vfill
		\end{frame}
	}


    \section{What is Programming?}

    \begin{frame}{What is Programming?}
        \begin{itemize}
            \item Process of giving instructions to a computer
            \item Computers follow step-by-step instructions precisely
            \item A program is a sequence of instructions that accomplishes a task
            \item Algorithms (logic) vs Programs (implemented in code)
        \end{itemize}
    \end{frame}


    \begin{frame}{History \& Evolution of Programming Languages}
        \begin{itemize}
            \item 1940s: Machine code (binary)
            \item 1950s: Assembly language (mnemonics)
            \item 1960s--70s: Structured languages (C, Pascal)
            \item 1990s: Object-oriented (Java, C++)
            \item 2000s--Now: Multi-paradigm, scripting (Python, JavaScript)
        \end{itemize}
    \end{frame}


    \begin{frame}{Flow of Program Development}
        \begin{enumerate}
            \item Define the problem
            \item Design algorithm and pseudocode or flowchart
            \item Write the program
            \item Compile/build and run
            \item Test and debug
            \item Update and maintain
        \end{enumerate}
    \end{frame}


    \begin{frame}{Program Structure}
        \begin{itemize}
            \item Typical structure of a program:
            \begin{itemize}
                \item Input (data from user or file)
                \item Processing (calculations, logic)
                \item Output (display results)
            \end{itemize}
            \item Uses variables, control flow (if/loops), and functions
        \end{itemize}
    \end{frame}


    \section{Types of Programming Languages}


    \begin{frame}{Levels of Programming Languages}
        \begin{itemize}
            \item Machine code (binary: 0s and 1s)
            \begin{itemize}
                \item The lowest-level representation of a program
                \item Directly executed by the CPU
            \end{itemize}
            
            \item Assembly language (x86 assembly, ARM assembly)
            \begin{itemize}
                \item Low-level language using mnemonics instead of binary
                \item Easier for humans to read than machine code
            \end{itemize}

            \item Higher-level languages (C, Python, Java etc.)
            \begin{itemize} 
                \item Abstract away hardware details
                \item Easier to write, read, and maintain
            \end{itemize}

            \item Domain-specific languages (DSL)
            \begin{itemize}
                \item Designed for a particular purpose or domain
                \item SQL for DB; Fortran for math, sciences and HPC; R for data science
            \end{itemize}
        \end{itemize}
    \end{frame}


    \begin{frame}{Pseudocode}
        \begin{itemize}
            \item Informal description of a program’s logic.
            \item Bridges the gap between human thought and actual code.
            \item Example:
            \begin{itemize}
                \item Start
                \item Input: integer a, integer b
                \item Let c = a + b
                \item Print c
                \item End
            \end{itemize}
        \end{itemize}
    \end{frame}


    \begin{frame}{Compiled vs Interpreted}
        \begin{itemize}
            \item \textbf{Compiled}: C, C++, Rust etc.
            \begin{itemize}
                \item Source code compiled into machine code by a compiler producing an execuatble file
                \item Faster execution
                \item Platform-specific
            \end{itemize}
            \item \textbf{Interpreted:} Python, Javascript etc.
            \begin{itemize}
                \item Code executed line by line by an interpreter
                \item No separate compilation step
                \item Easier to test and debug, but slower
            \end{itemize}
        \end{itemize}
    \end{frame}


    \begin{frame}{Programming Paradigms}
        \begin{itemize}
            \item \textbf{Procedural:} Step-by-step instructions (C)
            \item \textbf{Object-Oriented:} Organize into classes/objects (Java, Python)
            \item \textbf{Functional:} Emphasis on \emph{pure} functions, immutability (Haskell, Lisp)
        \end{itemize}
    \end{frame}


    \begin{frame}{Open Source vs Proprietary}
        \begin{itemize}
            \item \textbf{Open Source:} Free to use, modify, share (Python, GCC)
            \item \textbf{Proprietary:} Owned, restricted licenses (MATLAB)
            \item Ecosystem and libraries often drive language adoption
        \end{itemize}
    \end{frame}


    \begin{frame}{Performance vs Productivity}
        \begin{itemize}
            \item \textbf{Performance:} How fast a program runs (C, C++)
            \item \textbf{Productivity:} How quickly developers can write programs (Python, Ruby)
            \item Trade-off: High performance vs faster development
        \end{itemize}
    \end{frame}


    \begin{frame}{Portable Languages}
        \begin{itemize}
            \item Code can run on multiple platforms with little/no modification
            \item Example: Java -- Write Once, Run Anywhere
            \item Achieved via virtual machines or interpreters
        \end{itemize}
    \end{frame}


    \begin{frame}{Why So Many Languages?}
        \begin{itemize}
            \item Different goals:
            \begin{itemize}
                \item Performance: C, C++
                \item Safety and reliability: Go, Rust
                \item Ease of use and productivity: Python
                \item Domain focus: SQL, R
            \end{itemize}
            \item No single language is best for all problems.
        \end{itemize}
    \end{frame}


    \section{Development Tools}


    \begin{frame}{Development Tools: Editors \& IDEs}
        \begin{itemize}
            \item \textbf{Text Editors:} lightweight coding (Notepad++, Sublime)
            \item \textbf{IDEs:} full-featured, include debugging, auto-completion (VS Code, PyCharm)
        \end{itemize}
    \end{frame}


    \begin{frame}{Development Tools: Debugging, Profiling, VCS}
        \begin{itemize}
            \item \textbf{Debugger:} Step through code, inspect variables
            \item \textbf{Profiler:} Find performance bottlenecks
            \item \textbf{Version Control System, VCS:} Track changes, collaborate (Git)
        \end{itemize}
    \end{frame}


    \begin{frame}{Terminal and Command Line Interface (CLI)}
        \begin{itemize}
            \item \textbf{Terminal:} Text-based interface to interact with the OS
            \item \textbf{CLI:} Execute commands directly by typing them
            \item Used for compiling, running, and managing projects
            \item Common tools: \texttt{gcc}, \texttt{gdb}, \texttt{python}, \texttt{git}
        \end{itemize}
    \end{frame}


    \section*{Questions?}
\end{document}
