% \documentclass[12pt, handout]{beamer}
\documentclass[11pt, aspectratio=169]{beamer}

\input{header}


\title{Introduction to the C Language}
\author{Md. Aminul Islam Shazid}
\date{}


\begin{document}
    {
		\setbeamertemplate{footline}{}    % NO FOOTLINE FOR THESE TWO FRAMES
		\addtocounter{framenumber}{-2}    % not counting the title page and the outline in frame numbers

		\begin{frame}
			\titlepage
		\end{frame}

		\begin{frame}{Outline}
			\tableofcontents[subsectionstyle=hide]
		\end{frame}
	}

    \section{History}

    \begin{frame}{Creation of C}
        \begin{itemize}
            \item Created by Dennis Ritchie in close collaboration with Ken Thompson at the Bell Labs in the early 1970s
            \item The first operating system written in C is Unix which ran on the 16bit \href{https://en.wikipedia.org/wiki/PDP-11}{PDP-11} computer
            
        \end{itemize}
    \end{frame}
    
    \begin{frame}{Evolution of C}
        \begin{itemize}
            \item In 1978, Dennis Ritchie and Brian Kernighan released the book titled ``The C Programming Language"
            \item C was standardized by ANSI in 1989 and by ISO later in 1990, this is known as both C89 and C90
            \item The latest version of the standard is C23
        \end{itemize}
    \end{frame}

	\section{Features of C}
	
	\begin{frame}{Basic Features}
		\begin{itemize}
            \item High level (compared to assembly or machine code) 
            \item Also provides low level access
                \begin{itemize}
                    \item Allows writing inline assembly code
                    \item Provides direct access to memory management
                \end{itemize}
            \item Fewer keywords compared to other contemporary languages
        \end{itemize}
	\end{frame}

        \begin{frame}{Basic Features (cont.)}
            \begin{itemize}
                \item Compiled language, very fast
                \item Statically typed
                \item Not object oriented
                \item No garbage collection
            \end{itemize}
        \end{frame}

    \section*{Thank you.}

    % \appendix

    % \begingroup
    % \renewcommand{\section}[2]{}%
    % \begin{frame}[allowframebreaks]{References}
    %     \def\bibfont{\footnotesize}
    %     %\Wider{
    %         \bibliography{refdb.bib}
    %     %}
    % \end{frame}
    % \endgroup

    % \begin{frame}{Appendix}
    %     test appendix frame
    % \end{frame}

\end{document}
