\documentclass[12pt, aspectratio=169]{beamer}

\input{../header}

\title{Arrays and Strings in C}
\author{Md. Aminul Islam Shazid}
\date{}

\begin{document}

    {
		\setbeamertemplate{footline}{}    % NO FOOTLINE FOR THESE TWO FRAMES
		\addtocounter{framenumber}{-2}    % not counting the title page and the outline in frame numbers

		\begin{frame}
			\titlepage
		\end{frame}

		\begin{frame}{Outline}
            \vfill
			\tableofcontents[subsectionstyle=hide]
            \vfill
		\end{frame}
	}


\section{Introduction to Arrays}


    \begin{frame}{What is an Array?}
        \begin{itemize}
            \item An \textbf{array} is a collection of elements of the \emph{same} type stored in contiguous memory.
            \item Each element is accessed by an \textbf{index} (zero-based in C, meaning that the first element is at index 0): \texttt{arr[0]}, \texttt{arr[1]}, \dots
            \item Real-life analogy: a row of mailboxes, a list of student marks, or seats in a bus lined up.
            \item Arrays let us group related values under one name.
        \end{itemize}
    \end{frame}

    \begin{frame}[fragile]{1D Array: Declaration and Initialization}
        \textbf{Declaration:}
        \begin{verbatim}
int arr[5];              // declares an array of 5 integers
        \end{verbatim}

        \textbf{Declaration and Initialization:}
        \begin{verbatim}
int a[5] = {10, 20, 30, 40, 50};
int b[]  = {1, 2, 3};    // size inferred: 3
        \end{verbatim}
    \end{frame}

    \begin{frame}[fragile]{Accessing Individual Elements}
        \begin{itemize}
            \item Use square brackets with index: \texttt{arr[index]}.
            \item Example: \texttt{arr[2]} accesses the third element.
        \end{itemize}

        \textbf{Example: print elements}
        \begin{verbatim}
#include <stdio.h>

int main() {
    int a[5] = {10, 20, 30, 40, 50};
    printf("%d\n", a[2]);   // prints 30
    return 0;
}
        \end{verbatim}

        \textbf{Sample Output:}
        \begin{verbatim}
30
        \end{verbatim}
    \end{frame}

    \begin{frame}[fragile]{Take User Input Into an Array}
        \begin{verbatim}
#include <stdio.h>

int main() {
    int n, i;
    printf("How many numbers? ");
    scanf("%d", &n);
    int arr[100];    // assume max 100 for simplicity
    for (i = 0; i < n; i++) {
        scanf("%d", &arr[i]);
    }
    // print them
    for (i = 0; i < n; i++) {
        printf("arr[%d] = %d\n", i, arr[i]);
    }
    return 0;
}
        \end{verbatim}

        \textbf{Sample Input/Output:}
        \begin{verbatim}
How many numbers? 3
10 20 30
arr[0] = 10
arr[1] = 20
arr[2] = 30
        \end{verbatim}
    \end{frame}

    \begin{frame}[fragile]{Summing the Elements of an Array}
        \begin{verbatim}
#include <stdio.h>

int main() {
    int n, i, sum = 0;
    printf("n: ");
    scanf("%d", &n);
    int arr[100];
    for (i = 0; i < n; i++){
        scanf("%d", &arr[i]);
    }

    for (i = 0; i < n; i++){
        sum += arr[i];
    }
    printf("Sum = %d\n", sum);
}
        \end{verbatim}

        \textbf{Sample I/O:}
        \begin{verbatim}
n: 4
5 7 3 10
Sum = 25
Average = 6.25
        \end{verbatim}
    \end{frame}


    \section{Multidimensional Arrays}


    \begin{frame}{What is a Multidimensional Array?}
        \begin{itemize}
            \item A \textbf{2D array} is an array of arrays (like a table / matrix).
            \item Indices are written as \texttt{arr[row][col]}.
            \item Real-life analogies:
                \begin{itemize}
                    \item 2D: matricx or spreadsheet (rows and columns)
                    \item 3D: a stack of matrices
                \end{itemize}
            \item C supports arrays with any number of dimensions; common ones: 1D, 2D and 3D
        \end{itemize}
    \end{frame}

    \begin{frame}[fragile]{2D Array: Declaration and Initialization}
        \begin{verbatim}
int mat[3][4];               // 3 rows, 4 columns

int mat2[2][3] = {
    {1, 2, 3},
    {4, 5, 6}
};

// or flattened initialization:
int mat3[2][3] = {1,2,3,4,5,6};
        \end{verbatim}
    \end{frame}

    \begin{frame}[fragile]{Accessing Values in 2D Arrays}
        \begin{itemize}
            \item Access element at row r, column c by \texttt{mat[r][c]}.
            \item Example: \texttt{mat[1][2]} refers to second row, third column.
        \end{itemize}

        \textbf{Example: print a 2x3 matrix}
        \begin{verbatim}
#include <stdio.h>

int main() {
    int mat[2][3] = {{1,2,3},{4,5,6}};
    printf("%d\n", mat[1][2]); // prints 6
    return 0;
}
        \end{verbatim}

        \textbf{Sample Output:}
        \begin{verbatim}
6
        \end{verbatim}
    \end{frame}

    \begin{frame}[fragile]{Input and Print a 2D Array}
        \begin{verbatim}
#include <stdio.h>

int main() {
    int r = 2, c = 3, i, j, mat[2][3];
    printf("Please input a 2 by 3 matrix:\n");
    for (i = 0; i < r; i++)
        for (j = 0; j < c; j++)
            scanf("%d", &mat[i][j]);

    printf("\nYou entered:\n");
    for (i = 0; i < r; i++) {
        for (j = 0; j < c; j++)
            printf("%d ", mat[i][j]);
        printf("\n");
    }
}
        \end{verbatim}

        \textbf{Sample I/O:}
\begin{verbatim}
Please input a 2 by 3 matrix:
1 2 3
4 5 6

You entered:
1 2 3
4 5 6
\end{verbatim}
    \end{frame}

    \begin{frame}{3D and Higher Dimensions}
        \begin{itemize}
            \item A 3D array \texttt{int a[2][3][4];} can be thought of as 2 blocks, each block is a 3x4 matrix
            \item Real-life: for example, \textbf{day × row × column} measurements (temperature map over multiple days)
            \item Indexing: \texttt{a[day][row][col]}
        \end{itemize}
    \end{frame}


    \section{Strings in C}


    \begin{frame}{What is a String in C?}
        \begin{itemize}
            \item In C, a \textbf{string} is an array of \texttt{char} terminated by the null character \texttt{'\textbackslash0'}.
            \item Example: \texttt{char s[] = "hello";} actually creates 6 chars: \texttt{'h','e','l','l','o','\textbackslash0'}.
            \item Strings are manipulated through arrays and standard library functions in \texttt{<string.h>}.
            \item You can access individual characters with \texttt{s[i]}.
        \end{itemize}
    \end{frame}

    \begin{frame}[fragile]{Declare and Initialize Strings}
        \begin{verbatim}
char s1[] = "Hello";
char s2[10] = "Hi";      // remaining bytes unused (but available)
char s3[6] = {'H','i','!','\0'}; // explicit
        \end{verbatim}
    \end{frame}

    \begin{frame}[fragile]{Reading Strings from User}
        \begin{itemize}
            \item Avoid \texttt{gets()} (unsafe). Use \texttt{fgets()} or \texttt{scanf("\%s", ...)}.
            \item \texttt{scanf("\%s", s);} reads until whitespace — does not read spaces.
            \item \texttt{fgets(s, size, stdin);} reads a whole line (including spaces), but includes newline — may want to trim it.
        \end{itemize}

        \textbf{Example using fgets}
        \begin{verbatim}
#include <stdio.h>
#include <string.h>

int main() {
    char s[100];
    printf("Enter a line: ");
    fgets(s, sizeof(s), stdin);
    // remove trailing newline
    s[strcspn(s, "\n")] = '\0';
    printf("You wrote: \%s\n", s);
    return 0;
}
        \end{verbatim}
    \end{frame}

    \begin{frame}[fragile]{Common String Functions (from <string.h>)}
        \begin{itemize}
            \item \texttt{strlen(s)} — length of string (not counting '\textbackslash0')
            \item \texttt{strcmp(s1, s2)} — compare strings (0 if equal)
            \item \texttt{strcpy(dest, src)} — copy string
            \item \texttt{strcat(dest, src)} — concatenate
        \end{itemize}
    \end{frame}

    \begin{frame}[fragile]{Example: strcmp() and strcpy()}
        \begin{verbatim}
#include <stdio.h>
#include <string.h>

int main() {
    char a[20], b[20];
    strcpy(a, "apple");
    strcpy(b, "apple");
    if (strcmp(a, b) == 0)
        printf("Equal\n");
    else
        printf("Not equal\n");
    return 0;
}
        \end{verbatim}

        \textbf{Sample Output:}
        \begin{verbatim}
Equal
        \end{verbatim}
    \end{frame}

    \begin{frame}[fragile]{Upper-case and Lower-case Conversion (Manual)}
        \begin{verbatim}
#include <stdio.h>
#include <ctype.h>   // for toupper, tolower

int main() {
    char s[] = "Hello World!";
    int i = 0;
    // to upper
    while (s[i]) {
        s[i] = toupper((unsigned char)s[i]);
        i++;
    }
    printf("\%s\n", s);  // HELLO WORLD!
    return 0;
}
        \end{verbatim}

        \textbf{Notes:} you can use \texttt{tolower()} similarly. If you prefer not to use \texttt{ctype.h}, you can convert by arithmetic on ASCII when appropriate.
    \end{frame}

    \begin{frame}[fragile]{Example: Concatenate Strings (strcat)}
        \begin{verbatim}
#include <stdio.h>
#include <string.h>

int main() {
    char a[50] = "Hello";
    char b[] = " World";
    strcat(a, b);
    printf("\%s\n", a); // prints "Hello World"
    return 0;
}
        \end{verbatim}

        \textbf{Sample Output:}
        \begin{verbatim}
Hello World
        \end{verbatim}
    \end{frame}


    \section{Examples}

    \begin{frame}[fragile]{Example: Reverse an Array}
        \begin{verbatim}
#include <stdio.h>

int main() {
    int n = 5, i;
    int a[] = {1,2,3,4,5};
    for (i = 0; i < n/2; i++) {
        int tmp = a[i];
        a[i] = a[n-1-i];
        a[n-1-i] = tmp;
    }
    for (i = 0; i < n; i++) printf("%d ", a[i]);
    printf("\n");
    return 0;
}
        \end{verbatim}

        \textbf{Sample Output:}
        \begin{verbatim}
5 4 3 2 1
        \end{verbatim}
    \end{frame}

    \begin{frame}[fragile]{Example: Count Vowels in a String}
        \begin{verbatim}
#include <stdio.h>
#include <ctype.h>

int main() {
    char s[100];
    fgets(s, sizeof(s), stdin);
    s[strcspn(s, "\n")] = '\0';
    int i = 0, count = 0;
    while (s[i]) {
        char ch = tolower((unsigned char)s[i]);
        if (ch=='a'||ch=='e'||ch=='i'||ch=='o'||ch=='u'){
            count++;
        }
        i++;
    }
    printf("Vowels = %d\n", count);
    return 0;
}
        \end{verbatim}

        \textbf{Sample I/O:}
        \begin{verbatim}
Hello World
Vowels = 3
        \end{verbatim}
    \end{frame}


    \section{Summary}


    \begin{frame}{Summary}
        \begin{itemize}
            \item \textbf{Array:} contiguous collection of same-type elements, accessed by indices \texttt{arr[i]}.
            \item \textbf{1D/2D/3D:} use \texttt{arr[i]}, \texttt{arr[i][j]}, \texttt{arr[i][j][k]} respectively.
            \item \textbf{Strings:} arrays of \texttt{char} ending with \texttt{'\textbackslash0'}. Use \texttt{<string.h>} functions for convenience.
            \item \textbf{Input:} \texttt{scanf} or \texttt{fgets} (preferred for whole lines).
            \item \textbf{Memory:} array name often decays to pointer to the first element — but arrays are fixed-sized storage
        \end{itemize}
    \end{frame}


    \begin{frame}{Array Variable and Memory}
        \begin{itemize}
            \item \textbf{Important note:} In many contexts the array name (e.g., \texttt{arr}) \emph{decays} to a pointer (details on pointers in upcoming lectures) to the first element. Example: when passed to a function
            \item But \texttt{arr} itself is not a regular variable containing a value — you cannot reassign it (e.g., \texttt{arr = someOtherPointer;} is invalid)
            \item Internally arrays point to contiguous chunk of memory starting at the first element
        \end{itemize}
    \end{frame}



    \section{Exercises}


    \begin{frame}{Exercises}
        \begin{itemize}
            \item Write a program to read \texttt{n} integers into an array and print them in reverse order.
            \item Write a program to find the maximum and minimum values in an integer array.
            \item Write a program to remove duplicate elements from a small integer array (keep first occurrences).
            \item Write a program to multiply two 2x2 matrices and print the result.
            \item Write a program to read a line of text and print its length (without using \texttt{strlen}).
        \end{itemize}
    \end{frame}


    \begin{frame}{Exercises (cont.)}
        \begin{itemize}
            \item Write a program to check if a given string is a palindrome (ignore case and spaces).
            \item Write a program to count frequency of each digit (0-9) in an array of integers.
            \item Write a program to concatenate two strings without using \texttt{strcat}.
            \item Write a program to rotate the elements of an array to the right by \texttt{k} positions.
            \item Write a program to read a 3D array of size 2 × 2 × 2 and compute the sum of all elements.
        \end{itemize}
    \end{frame}


\end{document}
