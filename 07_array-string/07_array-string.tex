\documentclass[12pt, aspectratio=169]{beamer}

\usetheme{moloch}
\molochset{block=fill}

\usepackage{fontspec}
\setsansfont[
    UprightFont = Inter-Light,          % Use light as the normal font weight
    BoldFont = Inter-SemiBold,           % Use semibold for \textbf
    ItalicFont = Inter-LightItalic,      % Light italic for \textit
    BoldItalicFont = Inter-SemiBoldItalic % Semibold italic for \textbf with \textit
]{Inter}[RawFeature={+ss04, +ss03, +dlig, +tnum}]
% Inter font stylistic sets:
% ss01: alternate digits for 3, 4, 6, 8
% ss02: for disambiguation (with zero) in places like "Ill" or "O0" etc
% ss04: for disambiguation (without zero) in places like "Ill" etc
% ss03: round comma, quation marks

\usepackage{upquote}
\usepackage{microtype}
\UseMicrotypeSet[protrusion]{basicmath}    % disable protrusion for tt fonts

\usepackage{amsmath}
\usepackage{amssymb}
\usepackage{unicode-math}
\setmathfont{Erewhon Math}[Scale=1.14]

\usepackage{hyperref}
\pdfstringdefDisableCommands{\def\translate#1{#1}}
\usepackage{bookmark}
\usepackage{url}

\usepackage{natbib}
\usepackage{appendixnumberbeamer}
\usepackage{enumerate}
% \usepackage{enumitem}    % it is giving the error: TeX capacity exceeded
% \usepackage{footnotehyper}
\usepackage{graphicx}
\usepackage{caption}
% \usepackage{subcaption}
\usepackage{booktabs}
\usepackage{makecell}
\usepackage{array}
\newcolumntype{H}{>{\setbox0=\hbox\bgroup\let\pm\relax}c<{\egroup}@{}}
% \newcolumntype{H}{>{\setbox0=\hbox\bgroup}c<{\egroup}}% <--- removed @{}
% https://tex.stackexchange.com/questions/567724/can-i-hide-a-table-column-with-the-s-type-from-siunitx
% https://tex.stackexchange.com/questions/414143/hide-column-without-adding-whitespace-to-table


\definecolor{airforceblue}{rgb}{0.36, 0.54, 0.66}
\hypersetup{
    colorlinks=true,
    linkcolor={mDarkTeal},    % this colour is defined by the moloch theme
    filecolor={Maroon},
    citecolor={airforceblue!120},
    urlcolor={airforceblue!140},
    pdfcreator={xelatex},
    bookmarksopen=true,    % Expand bookmarks in the PDF
    bookmarksnumbered=true % Include numbering in bookmarks
}

\bibliographystyle{apalike}

\let\oldcite=\cite
\renewcommand{\cite}[1]{\textcolor{airforceblue!120}{\oldcite{#1}}}
\let\oldcitet=\citet
\renewcommand{\citet}[1]{\textcolor{airforceblue!120}{\oldcitet{#1}}}
\let\oldcitep=\citep
\renewcommand{\citep}[1]{\textcolor{airforceblue!120}{\oldcitep{#1}}}


% \setlength{\leftmargini}{0em}
\setbeamercolor{page number in head/foot}{fg=gray}
\setbeamertemplate{footline}[frame number]
\setbeamertemplate{itemize items}[circle]
\setbeamertemplate{enumerate items}[circle]
\setbeamertemplate{sections/subsections in toc}[circle]
\setbeamertemplate{frametitle continuation}[from second][(cont.)]
\setbeamercovered{transparent}
\beamertemplatenavigationsymbolsempty


\AtBeginSubsection[]{
    {
        \begin{frame}[noframenumbering, plain]
            \subsectionpage
        \end{frame}
    }
}


\newcommand\Wider[2][4em]{%
    \makebox[\linewidth][c]{%
        \begin{minipage}{\dimexpr\textwidth+#1\relax}
            % \raggedright#2
            \centering#2
        \end{minipage}%
    }%
}

% \newenvironment{myitemize}{
%     \begin{itemize}
%         \vspace{1em}
%         \setlength{\itemsep}{0.7\baselineskip}
% }{
%         \vspace{1em}
%     \end{itemize}
% }

% \newenvironment{myenumerate}{
%     \begin{enumerate}
%         \vspace{1em}
%         \setlength{\itemsep}{0.7\baselineskip}
% }{
%         \vspace{1em}
%     \end{enumerate}
% }

\title{Arrays and Strings in C}
\author{Md. Aminul Islam Shazid}
\date{}

\begin{document}

    {
		\setbeamertemplate{footline}{}    % NO FOOTLINE FOR THESE TWO FRAMES
		\addtocounter{framenumber}{-2}    % not counting the title page and the outline in frame numbers

		\begin{frame}
			\titlepage
		\end{frame}

		\begin{frame}{Outline}
            \vfill
			\tableofcontents[subsectionstyle=hide]
            \vfill
		\end{frame}
	}


\section{Introduction to Arrays}


    \begin{frame}{What is an Array?}
        \begin{itemize}
            \item An \textbf{array} is a collection of elements of the \emph{same} type stored in contiguous memory.
            \item Each element is accessed by an \textbf{index} (zero-based in C, meaning that the first element is at index 0): \texttt{arr[0]}, \texttt{arr[1]}, \dots
            \item Example: list of student marks
            \item Arrays let us group related values under one name
        \end{itemize}
    \end{frame}

    \begin{frame}[fragile]{1D Array: Declaration and Initialization}
        \textbf{Declaration:}
        \begin{verbatim}
int arr[5];              // declares an array of 5 integers
        \end{verbatim}

        \textbf{Declaration and Initialization:}
        \begin{verbatim}
int a[5] = {10, 20, 30, 40, 50};
int b[]  = {1, 2, 3};    // size inferred: 3
        \end{verbatim}
    \end{frame}

    \begin{frame}{Accessing Individual Elements}
        \begin{itemize}
            \item Use square brackets with index: \texttt{arr[index]}.
            \item Example: \texttt{arr[2]} accesses the third element.
        \end{itemize}

        \textbf{Example: print elements}
        \inputminted[
            frame=none,
            bgcolor=gray!20,
            linenos, 
            breaklines=true,
            fontsize=\fontsize{12pt}{13pt}\selectfont,
            % firstline=1,
            % lastline=7,
            % firstnumber=1
        ]{c}{../code-examples/07_01_accessing-array-element.c}
    \end{frame}

    \begin{frame}{Example: Take User Input Into an Array}
        \inputminted[
            frame=none,
            bgcolor=gray!20,
            linenos, 
            breaklines=true,
            fontsize=\fontsize{12pt}{13pt}\selectfont,
        ]{c}{../code-examples/07_02_array-io.c}
    \end{frame}


    \begin{frame}{Example: Summing the Elements of an Array}
        \inputminted[
            frame=none,
            bgcolor=gray!20,
            linenos, 
            breaklines=true,
            fontsize=\fontsize{12pt}{13pt}\selectfont,
        ]{c}{../code-examples/07_03_sum-array-values.c}
    \end{frame}


    % \begin{frame}[fragile]{Example: Reverse an Array}
    %     \inputminted[
    %         frame=none,
    %         bgcolor=gray!20,
    %         linenos, 
    %         breaklines=true,
    %         fontsize=\fontsize{12pt}{13pt}\selectfont,
    %     ]{c}{../code-examples/07_10_array-reverse.c}
    % \end{frame}


    \section{Multidimensional Arrays}


    \begin{frame}{Multidimensional Array}
        \begin{itemize}
            \item An array with more than one dimension
            \item Real-life analogies:
                \begin{itemize}
                    \item 2D: matricx or spreadsheet (rows and columns), indices are written as \texttt{arr[row][col]}
                    \item 3D: a stack of matrices
                \end{itemize}
            \item C supports arrays with any number of dimensions
        \end{itemize}
    \end{frame}

    \begin{frame}[fragile]{2D Array: Declaration and Initialization}
        \begin{verbatim}
int mat[3][4];               // 3 rows, 4 columns

int mat2[2][3] = {
    {1, 2, 3},
    {4, 5, 6}
};

// or flattened initialization:
int mat3[2][3] = {1,2,3,4,5,6};
        \end{verbatim}
    \end{frame}


    \begin{frame}{Accessing Values in 2D Arrays}
        \begin{itemize}
            \item Access element at row r, column c by \texttt{mat[r][c]}
            \item Example: \texttt{mat[1][2]} refers to second row, third column
        \end{itemize}

        \vspace{1em}
        
        \textbf{Example: print a 2x3 matrix}
        \inputminted[
            frame=none,
            bgcolor=gray!20,
            linenos, 
            breaklines=true,
            fontsize=\fontsize{12pt}{13pt}\selectfont,
        ]{c}{../code-examples/07_04_accessing-2d-array-element.c}
    \end{frame}


    \begin{frame}{Example: Input and Print a 2D Array}
        \inputminted[
            frame=none,
            bgcolor=gray!20,
            linenos, 
            breaklines=true,
            fontsize=\fontsize{12pt}{13pt}\selectfont,
            firstline=1,
            lastline=14,
            firstnumber=1
        ]{c}{../code-examples/07_05_2d-array-io.c}

        \textit{Continued in next slide}
    \end{frame}


    \begin{frame}{Example: Input and Print a 2D Array (cont.)}
        \inputminted[
            frame=none,
            bgcolor=gray!20,
            linenos, 
            breaklines=true,
            fontsize=\fontsize{12pt}{13pt}\selectfont,
            firstline=14,
            lastline=24,
            firstnumber=14
        ]{c}{../code-examples/07_05_2d-array-io.c}
    \end{frame}


    \begin{frame}{3D and Higher Dimensions}
        \begin{itemize}
            \item A 3D array \texttt{int arr[2][3][4];} can be thought of as 2 blocks, each block is a 3x4 matrix
            \item Real-life: for example, \textbf{block × row × column} measurements (temperature map over multiple days)
            \item Indexing: \texttt{a[block][row][col]}
        \end{itemize}
    \end{frame}


    \section{Strings in C}

    \begin{frame}{What is a String in C?}
        \begin{itemize}
            \item In C, a \textbf{string} is an array of \texttt{char} terminated by the null character \texttt{'\textbackslash0'}
            \item Example: \texttt{char s[] = "hello";} actually creates 6 chars: \texttt{'h','e','l','l','o','\textbackslash0'}
            \item Strings are manipulated through arrays and standard library functions in \texttt{<string.h>}
            \item You can access individual characters with \texttt{s[i]}
        \end{itemize}
    \end{frame}


    \begin{frame}[fragile]{Declare and Initialize Strings}
        \begin{verbatim}
char s1[] = "Hello";
char s2[10] = "Hi";      // remaining bytes unused
char s3[6] = {'H','i','!','\0'}; // explicit
        \end{verbatim}
    \end{frame}


    \begin{frame}[fragile]{Reading Strings from User}
        \begin{itemize}
            \item Avoid \texttt{gets()} (unsafe). Use \texttt{fgets()} or \texttt{scanf("\%s", ...)}
            \item \texttt{scanf("\%s", s);} reads until whitespace, does not read spaces
            \item \texttt{fgets(s, size, stdin);} reads a whole line (including spaces)
            \item However, the string from \texttt{fgets()} includes newline (\verb|\n|)
            \item May want to trim this newline (trimming it, is often not required, depends on use case)
        \end{itemize}
    \end{frame}


    \begin{frame}{Example: String Input Using \texttt{scanf()}}
        \inputminted[
            frame=none,
            bgcolor=gray!20,
            linenos, 
            breaklines=true,
            fontsize=\fontsize{12pt}{13pt}\selectfont
        ]{c}{../code-examples/04_01_greetings.c}

        Note that in \texttt{scanf()}, we provided \texttt{username} as the second argument and not \texttt{\&username}.
        
        This is because \texttt{username} itself holds the memory address of the character array.

        Recall that the \texttt{\&var} returns the memory address of the variable named \texttt{var}.
    \end{frame}


    \begin{frame}{Example: String Input Using \texttt{fgets()}}
        \inputminted[
            frame=none,
            bgcolor=gray!20,
            linenos, 
            breaklines=true,
            fontsize=\fontsize{12pt}{13pt}\selectfont
        ]{c}{../code-examples/07_11_fgets-string_no-trim.c}
    \end{frame}


    \begin{frame}{Example: String Input Using \texttt{fgets()} (cont.)}
        \inputminted[
            frame=none,
            bgcolor=gray!20,
            linenos, 
            breaklines=true,
            fontsize=\fontsize{12pt}{13pt}\selectfont
        ]{c}{../code-examples/07_06_fgets-string.c}
    \end{frame}


    \begin{frame}{Example: String Input Using \texttt{fgets()} (cont.)}
        \inputminted[
            frame=none,
            bgcolor=gray!20,
            linenos, 
            breaklines=true,
            fontsize=\fontsize{12pt}{13pt}\selectfont,
            firstline=9,
            lastline=11,
            firstnumber=9
        ]{c}{../code-examples/07_06_fgets-string.c}

        In the above code, \texttt{strcspn(s, "\textbackslash n")} searches for and returns the index of the newline character (\texttt{"\textbackslash n"}) in the string named \texttt{s}.\\

        \texttt{s[strcspn(s, "\textbackslash n")] = '\textbackslash 0';} replaces the newline character with the null terminator character (\texttt{'\textbackslash 0'}).
    \end{frame}


    \begin{frame}{Common String Functions (from <string.h>)}
        \begin{itemize}
            \item \texttt{strlen(s)} — length of string (not counting '\textbackslash0')
            \item \texttt{strcmp(s1, s2)} — compare strings (returns 0 if equal)
            \item \texttt{strcpy(dest, src)} — copy string
            \item \texttt{strcat(dest, src)} — concatenate
        \end{itemize}
    \end{frame}

    
    \begin{frame}{Example: Comparing Two Strings: \texttt{strcmp()}}
        \inputminted[
            frame=none,
            bgcolor=gray!20,
            linenos, 
            breaklines=true,
            fontsize=\fontsize{12pt}{13pt}\selectfont
        ]{c}{../code-examples/07_07_strcmp.c}

        \texttt{strcmp()} returns 0 if the two strings are same.
    \end{frame}


    \begin{frame}{Example: Copying a String: \texttt{strcpy()}}
        \inputminted[
            frame=none,
            bgcolor=gray!20,
            linenos, 
            breaklines=true,
            fontsize=\fontsize{12pt}{13pt}\selectfont
        ]{c}{../code-examples/07_08_strcpy.c}
    \end{frame}


%     \begin{frame}[fragile]{Example: Upper-case and Lower-case Conversion (Manual)}
%         \begin{verbatim}
% #include <stdio.h>
% #include <ctype.h>   // for toupper, tolower

% int main() {
%     char s[] = "Hello World!";
%     int i = 0;
%     // to upper
%     while (s[i]) {
%         s[i] = toupper((unsigned char)s[i]);
%         i++;
%     }
%     printf("\%s\n", s);  // HELLO WORLD!
%     return 0;
% }
%         \end{verbatim}

%         \textbf{Notes:} you can use \texttt{tolower()} similarly. If you prefer not to use \texttt{ctype.h}, you can convert by arithmetic on ASCII when appropriate.
%     \end{frame}




%     \begin{frame}[fragile]{Example: Count Vowels in a String}
%         \begin{verbatim}
% #include <stdio.h>
% #include <ctype.h>

% int main() {
%     char s[100];
%     fgets(s, sizeof(s), stdin);
%     s[strcspn(s, "\n")] = '\0';
%     int i = 0, count = 0;
%     while (s[i]) {
%         char ch = tolower((unsigned char)s[i]);
%         if (ch=='a'||ch=='e'||ch=='i'||ch=='o'||ch=='u'){
%             count++;
%         }
%         i++;
%     }
%     printf("Vowels = %d\n", count);
%     return 0;
% }
%         \end{verbatim}
%     \end{frame}


    \begin{frame}[fragile]{Example: Concatenate Strings (strcat)}
        \inputminted[
            frame=none,
            bgcolor=gray!20,
            linenos, 
            breaklines=true,
            fontsize=\fontsize{12pt}{13pt}\selectfont
        ]{c}{../code-examples/07_09_strcat.c}
    \end{frame}


    \section{Summary}


    \begin{frame}{Summary}
        \begin{itemize}
            \item \textbf{Array:} contiguous collection of same-type elements, accessed by indices \texttt{arr[i]}
            \item \textbf{1D/2D/3D:} use \texttt{arr[i]}, \texttt{arr[i][j]}, \texttt{arr[i][j][k]} respectively
            \item \textbf{Strings:} arrays of \texttt{char} ending with \texttt{'\textbackslash0'}
            \item \texttt{<string.h>} contains functions that operate on strings
            \item \textbf{Input:} \texttt{scanf} or \texttt{fgets} (preferred for whole lines)
        \end{itemize}
    \end{frame}


    \begin{frame}{Array Variable and Memory}
        \begin{itemize}
            \item \textbf{Important note:} In many contexts (for example, when passing to a function), the array name (say, \texttt{arr}) is a pointer to the first element (details on pointers in upcoming lectures)
            \item Meaning that, \texttt{arr} actually holds the memory address in which the first element of the array is stored
            \item To summarize: the array name points to a contiguous block of memory starting at the first element of the array
        \end{itemize}
    \end{frame}



    \section{Exercises}


    \begin{frame}{Exercises}
        \begin{itemize}
            \item Write a program to read \texttt{n} integers into an array and print them in reverse order
            \item Write a program to find the maximum and minimum values in an integer array
            \item Write a program to remove duplicate elements from a small integer array (keep first occurrences)
            \item Write a program to multiply two 2x2 matrices and print the result
            \item Write a program to read a line of text and print its length (without using \texttt{strlen})
        \end{itemize}
    \end{frame}


    \begin{frame}{Exercises (cont.)}
        \begin{itemize}
            \item Write a program to check if a given string is a palindrome (ignore case and spaces)
            \item Write a program to count frequency of each digit (0-9) in an array of integers
            \item Write a program to concatenate two strings without using \texttt{strcat}
            \item Write a program to rotate the elements of an array to the right by \texttt{k} positions
            \item Write a program to read a 3D array of size 2 × 2 × 2 and compute the sum of all elements
        \end{itemize}
    \end{frame}


\end{document}
