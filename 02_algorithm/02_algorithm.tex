% \documentclass[12pt, handout]{beamer}
\documentclass[12pt, aspectratio=169]{beamer}

\input{../header}
\input{../header_tikz.tex}


\title{Algorithm, Pseudocode and Flowchart}
\author{Md. Aminul Islam Shazid}
\date{}


\begin{document}
    {
		\setbeamertemplate{footline}{}    % NO FOOTLINE FOR THESE TWO FRAMES
		\addtocounter{framenumber}{-2}    % not counting the title page and the outline in frame numbers

		\begin{frame}
			\titlepage
		\end{frame}

		\begin{frame}{Outline}
            \vfill
            \small
			\tableofcontents[subsectionstyle=hide]
            \vfill
		\end{frame}
	}


    \section{Introduction}
    \begin{frame}{Introduction}
        \begin{itemize}
            \item Algorithms, flowcharts, and pseudocode are essential tools for problem-solving
            \item They provide a bridge between problem analysis and actual programming
            \item This lecture introduces their concepts, notations, and best practices
        \end{itemize}
    \end{frame}


    \section{Algorithms}


    \begin{frame}{What is an Algorithm?}
        \begin{itemize}
            \item A step-by-step procedure to solve a problem
            \item Unambiguous and finite sequence of instructions
            \item Example: A recipe for cooking is an algorithm in real life
        \end{itemize}
    \end{frame}


    \begin{frame}{Characteristics of a Good Algorithm}
        \begin{itemize}
            \item Finiteness: must terminate after finite steps
            \item Definiteness: each step is clearly defined
            \item Input: specified set of inputs
            \item Output: specified set of outputs
            \item Effectiveness: steps can be performed with available resources
        \end{itemize}
    \end{frame}


    \begin{frame}{Examples of Simple Algorithms}
        \begin{itemize}
            \item Finding the maximum of three numbers
            \item Calculating factorial of a number
            \item Linear search in an array
        \end{itemize}
    \end{frame}


    \begin{frame}{Example Algorithm: Finding the Area of a Triangle}
        \begin{enumerate}
            \item Input base, $b$ and height, $h$
            \item Let area, $a = bh/2$
            \item Output $a$
        \end{enumerate}
    \end{frame}


    \section{Control Structures}

    \begin{frame}{Control Flow and Structure of a Program}
        \begin{itemize}
            \item Need to be familiar with control structure to be able to write algorithms
            \item Control flow or control structure can be divided into a few types:
            \begin{itemize}
                \item \textbf{Sequence}: step by step execution of commands from top to bottom
                \item \textbf{Selection} or \textbf{conditional execution}: exceuting a codeblock if certain conditions are met (if-else statement)
                \item \textbf{Iteration} or \textbf{loop}: repeatedly executing a block of code or commands while a certain condition is true, stop the loop if the codition is no longer true
            \end{itemize}
            \item Every program can be built using the three structures
        \end{itemize}

        \vspace{1em}

        Additionally, functions (collection of commands) with zero or more inputs can be defined.
    \end{frame}


    \begin{frame}{Sequence}
        \begin{itemize}
            \item Code or commands are executed step by step, or sequentially
            \item Example: Input a number, then calculate its square, then print the result
        \end{itemize}
    \end{frame}


    \begin{frame}{Example Algorithm: Showing the Square of a Number}
        \begin{enumerate}
            \item Input a number, $a$
            \item Let square, $s = a^2$
            \item Output $s$
        \end{enumerate}
        In the above, the commands are executed from top to bottom sequentially.
    \end{frame}


    \begin{frame}{Selection or Conditional Execution}
        \begin{itemize}
            \item \textbf{IF}: execute a block if condition is true
            \item \textbf{IF-ELSE}: choose between two alternatives
            \item \textbf{ELSE IF ladder}: multiple conditions
            \item Can have an \textbf{IF} statement inside another, this is called nested \textbf{IF} statements
        \end{itemize}
    \end{frame}


    \begin{frame}{Example Algorithm: Finding the Larger of Two Numbers}
        \begin{enumerate}
            \item Input two numbers, $a$ and $b$
            \item If $a$ is larger than b:
            \begin{enumerate}[i]
                \item Then, output a
                \item Else, output b
            \end{enumerate}
        \end{enumerate}
    \end{frame}


    \begin{frame}{Iteration or Loops}
        \begin{itemize}
            \item \textbf{FOR loop}: repeatedly execute commands for a fixed number of times
            \item \textbf{WHILE loop}: repeatedly execute a block of code while a condition is true. Usually, the number of iteration required until the condition becomes false, is not in known advanced
            \item \textbf{DO-WHILE loop}: run the commands at least once, then repeat if condition holds
            \item Can have a loop inside another loop, it is known as nested looping
        \end{itemize}
        For the purpose of this slide, only \textbf{WHILE} loop shall be used to keep things simple for now.
    \end{frame}


    \begin{frame}{Example Algorithm: Outputting the First $n$ Integers}
        \begin{enumerate}
            \item Input $n$
            \item Set $i=1$
            \item While $i<=n$:
            \begin{enumerate}
                \item Output $i$
                \item $i=i+1$
            \end{enumerate}
        \end{enumerate}
    \end{frame}


    % \begin{frame}{While vs Do-While}
    %     \begin{columns}
    %     \begin{column}{0.5\textwidth}
    %         \begin{tikzpicture}
    %             \node (start) [startstop] {Start};
    %             \node (cond) [decision, below of=start, yshift=-1cm] {Condition};
    %             \node (command) [process, below of=cond, yshift=-1cm] {Perform commands};
    %             \node (stop) [startstop, below of=command, yshift=-1cm] {Stop};

    %             \draw [arrow] (start) -- (cond);
    %             \draw [arrow] (cond) -- (command) node[midway, right] {True};
    %             \draw [arrow] (command.west) -- ++(-1, 0) |- (cond);
    %             \draw [arrow] (cond.east) -- node[midway, above] {False} ++(1.5, 0) |- (stop);
    %         \end{tikzpicture}
    %     \end{column}
    %     \begin{column}{0.5\textwidth}
    %         \begin{tikzpicture}
    %             \node (start) [startstop] {Start};
    %             \node (command) [process, below of=start, yshift=-1cm] {Perform commands};
    %             \node (cond) [decision, below of=command, yshift=-1cm] {Condition};
    %             \node (stop) [startstop, below of=cond, yshift=-2cm] {Stop};

    %             \draw [arrow] (start) -- (command);
    %             \draw [arrow] (command) -- (cond);
    %             \draw [arrow] (cond.east) -- ++(1.5, 0) node[midway, above] {True} |- (command.east);
    %             \draw [arrow] (cond) -- (stop) node[midway, right] {False};
    %         \end{tikzpicture}
    %     \end{column}
    %     \end{columns}
    % \end{frame}


    \begin{frame}{Break and Continue}
        \begin{itemize}
            \item \textbf{BREAK}: exit a loop immediately without any further iteration. When inside nested loops, it exits out of the loop in which the \textbf{BREAK} statement is called
            \item \textbf{CONTINUE}: skip the rest of the current iteration, proceed to the next iteration
        \end{itemize}
    \end{frame}


    \begin{frame}{Functions}
        \begin{itemize}
            \item Collection of commands that perform a specific task
            \item Groups together logic or commands that needs to be written across multiple places in a program
            \item Usually given a name
            \item Can call a function with its name followed by its parameters in brackets
            \item Can have zero or more inputs. These inputs are known as parameter or arguments
            \item Since a function is only defined once and subsequently called only using its name, this reduces code duplication leading to better readability and maintainability of code
        \end{itemize}
    \end{frame}


    \begin{frame}{Recursion}
        \begin{itemize}
            \item Function calling itself to solve smaller subproblems
            \item Example: factorial, Fibonacci
            \item Must have a base case to terminate
            \item A base case is a condition which when true, the function stops calling itself and returns the final result
            \item The function must be able to reach its base case, otherwise it will turn into an infinite loop
        \end{itemize}
    \end{frame}


    \begin{frame}{Before Designing an Algorithm}
        Before writing an algorithm, think carefully about the following:

        \begin{itemize}
            \item \textbf{Inputs}: What data is required to solve the problem?
            \item \textbf{Outputs}: What results should be shown?
            \item \textbf{Variables}: What values need to be stored and updated during execution?
            \item \textbf{Processing steps}: What operations or calculations are required?
            \item \textbf{Formulas}: What mathematical or logical formulas are needed?
            \item \textbf{Decision making}: Are conditional checks (IF-ELSE) required?
            \item \textbf{Repetition}: Are loops required, and should the output be shown once or repeatedly?
            \item \textbf{Loop control}: What condition starts and stops each loop?
            \item \textbf{Recursion}: If recursion is used, what is the base case and how does the problem reduce?
        \end{itemize}
    \end{frame}



    \begin{frame}{Example Algorithm: Factorial of a Number}
        \begin{enumerate}
            \item Input an integer, n
            \item Set result = 1
            \item While n is larger than 1, repeat the following:
            \begin{enumerate}[i]
                \item result = result × n
                \item n = n - 1
            \end{enumerate}
            \item Output the result
        \end{enumerate}
        
        \vspace{1em}

        Note: In step 3, ``While'' is a looping construct.\\
        
        The statements under the ``While'' key-word are executed repeatedly as long as the condition (n is larger than 1) is true.
    \end{frame}


    \section{Flowcharts}


    \begin{frame}{Definition and Purpose}
        \begin{itemize}
            \item Flowchart: graphical representation of an algorithm
            \item Uses standard symbols to show the flow of control
            \item Helps visualize program logic before coding
        \end{itemize}
    \end{frame}


    \begin{frame}{Flowchart Shapes}
        \begin{columns}
        \begin{column}{0.5\textwidth}
            \begin{itemize}
            \item \textbf{Start/Stop}: ellipse
            \item \textbf{Process}: rectangle
            \item \textbf{Decision}: diamond
            \item \textbf{Input/Output}: parallelogram
            \item \textbf{Sequence}: arrow
        \end{itemize}
        \end{column}
        \begin{column}{0.5\textwidth}
            \begin{tikzpicture}
                \node (startstop) [startstop] {Start/stop};
                \node (process) [process, right of=startstop, xshift=2.5cm] {Process};
                \node (cond) [decision, below of=startstop, yshift=-2cm] {Decision};
                \node (io) [io, below of=process, xshift=.35cm, yshift=-2cm] {Input/output};
                \draw [arrow] (0,-5) -- (4,-5) node[midway, above] {Sequence};
            \end{tikzpicture}
        \end{column}
        \end{columns}
    \end{frame}


    \begin{frame}{Example Flowchart: Factorial of a Number}
        \centering
        \begin{tikzpicture}%[node distance=2cm]
            \node (start) [startstop] {Start};
            \node (input) [io, right of=start, xshift=2cm] {Input n};
            \node (init_res) [process, right of=input, xshift=2cm] {Set result=1};
            \node (check_n) [decision, below of=init_res, yshift=-1cm] {n$>$1?};
            \node (upd_res) [process, left of=check_n, xshift=-3cm] {result = result × n};
            \node (upd_n) [process, left of=upd_res, xshift=-2.5cm] {n = n -1};
            \node (output) [io, right of=check_n, xshift=2.5cm] {Output result};
            \node (stop) [startstop, below of=output, yshift=-1cm] {Stop};

            \draw [arrow] (start) -- (input);
            \draw [arrow] (input) -- (init_res);
            \draw [arrow] (init_res) -- (check_n);
            \draw [arrow] (check_n) -- (upd_res) node[midway, above] {Yes};
            \draw [arrow] (upd_res) -- (upd_n);
            \draw [arrow] (upd_n.south) -- ++(0, -1) -| (check_n.south);
            \draw [arrow] (check_n) -- (output) node[midway, above] {No};
            \draw [arrow] (output) -- (stop);
        \end{tikzpicture}
    \end{frame}


    \section{Pseudocode}


    \begin{frame}{Purpose of Pseudocode}
        \begin{itemize}
            \item Represents algorithms in structured, human-readable code
            \item Independent of programming language, but may include programming key-words
            \item Easier to understand and refine before coding
        \end{itemize}
    \end{frame}


    \begin{frame}{Conventions}
        \begin{itemize}
            \item Use natural language mixed with structured logic
            \item Variable names should be consistent and meaningful
            \item Keywords like Input, If, While, For, Output, Function
            \item Indentation to show block structure
            \item Colons indicate the beginning of a block
            \item Keywords like EndIf, EndWhile, EndFor, EndFunction to indicate the end of a code block
            \item Pseudocode should be language-independent
        \end{itemize}
    \end{frame}


    \begin{frame}[fragile]{Example pseudocode: Area of a Triangle}
        \begin{verbatim}
Start
Input base, height
Set area = base * height / 2
Output area
End
        \end{verbatim}    
    \end{frame}


    \begin{frame}[fragile]{Example pseudocode: Factorial of a Number}
        \begin{verbatim}
Start
Input n
Set result = 1
While n>1:
    result = result * n
    n = n - 1
EndWhile
Output result
End
        \end{verbatim}
    \end{frame}


    \begin{frame}[fragile]{Example Pseudocode: Function for Finding Factorial}
        The following defines a function named \texttt{Factorial()} with a single input \texttt{n}.
        \begin{verbatim}
Function Factorial(n):
    While n>1:
        result = result * n
        n = n - 1
    EndWhile
    Return result
EndFunction
        \end{verbatim}
        The \texttt{Return} keyword indicates which value to return to the caller of the function. It also marks the end of execution of a function. 
    \end{frame}


    \begin{frame}[fragile]{Example Pseudocode: Factorial of a Number using Recursion}
        A recursive function (function that calls itself) named \verb|Factorial()| is defined that takes a single input:\\

        \begin{verbatim}
Function Factorial(n):
    If (n==0):
        Return 1
    Else:
        Return n * Factorial(n-1)
    EndIf
EndFunction
        \end{verbatim}

        Note: ``a==b" checks whether a is equal to b, returns True if they are equal, otherwise, returns False.
    \end{frame}


    \begin{frame}{Best Practices}
    \begin{itemize}
        \item Keep flowcharts clean and uncluttered
        \item Use consistent symbols and indentation
        \item Pseudocode should be language-independent
        \item Algorithms should be logically ordered and unambiguous
    \end{itemize}
    \end{frame}
    

    \begin{frame}{Common Pitfalls}
    \begin{itemize}
        \item Overcomplicating flowcharts with too many details
        \item Ambiguous pseudocode (mixing multiple languages)
        \item Ignoring edge cases in algorithms
        \item Writing unstructured logic
    \end{itemize}
    \end{frame}

    
    \section{Examples}

    \begin{frame}{Putting It All Together}
        \begin{block}{Example task}
            Compute the sum of all even numbers from 1 to N
        \end{block}
    \end{frame}


    \begin{frame}{Algorithm}
    \begin{enumerate}
        \item Read n
        \item Set sum = 0, i = 1
        \item While i <= n:
            \begin{enumerate}[i]
                \item If i is even:
                \begin{itemize}
                    \item add i to sum
                \end{itemize}
                \item Add 1 to i
            \end{enumerate}
        \item Print sum
    \end{enumerate}
    \end{frame}


    \begin{frame}{Flowchart}
        \centering
        \begin{tikzpicture}[node distance=4cm]
            \node (start) [startstop] {Start};
            \node (in) [io, right of=start] {Input N};
            \node (init_sum_i) [process, right of=in] {Sum = 0, i = 1};
            \node (i_lt_n) [decision, below of=init_sum_i, yshift=-0cm] {i $\leq$ N?};
            \node (i_even) [decision, left of=i_lt_n] {i even?};
            \node (upd_sum) [process, left of=i_even, align=center] {Sum = Sum + i};
            \node (upd_i) [process, above of=i_even, yshift=-2cm] {i = i + 1};
            \node (out) [io, below of=i_lt_n, yshift=2cm] {Print Sum};
            \node (stop) [startstop, left of=out] {Stop};
            
            \draw [arrow] (start.east) -- (in.west);
            \draw [arrow] (in.east) -- (init_sum_i.west);
            \draw [arrow] (init_sum_i.east) -- ++(1,0) |- (i_lt_n.east);
            \draw [arrow] (i_lt_n.west) -- (i_even.east) node[midway,above] {Yes};
            \draw [arrow] (i_even.west) -- (upd_sum.east) node[midway,above] {Yes};
            \draw [arrow] (i_even.north) -- (upd_i.south) node[midway,right] {No};
            \draw [arrow] (upd_sum.north) |- (upd_i.west);
            \draw [arrow] (upd_i.east) -| (i_lt_n.north);
            \draw [arrow] (i_lt_n.south) -- (out.north) node[midway,right] {No};
            \draw [arrow] (out) -- (stop);
        \end{tikzpicture}
    \end{frame}


    \begin{frame}[fragile]{Pseudocode}
        \begin{verbatim}
Start
Input n
sum = 0
i = 1
While (i <= n):
    If (i is even):
        sum = sum + i 
    EndIf
    i = i + 1
EndWhile
Output sum
End
        \end{verbatim}
    \end{frame}


    \begin{frame}[fragile]{Example: Find Whether a Number is Even or Odd}
        \begin{verbatim}
Start
Input num
If (num mod 2 == 0):
    Output Even
Else
    Output Odd
EndIf
End
        \end{verbatim}

        Note: In the above, $x \, \mathrm{mod} \, y$ returns the remainder when $x$ is divided by $y$.
    \end{frame}


    \begin{frame}[fragile]{Example: Solution of Quadratic Equation}
        The equation is given as: $ax^2 + bx + c = 0$
        \begin{verbatim}
Start
Input a, b, c
x1 = (-b + sqrt(b^2 - 4ac)) / 2a
x2 = (-b - sqrt(b^2 - 4ac)) / 2a
Output x1, x2
End
        \end{verbatim}

        Note: In the above, \verb|sqrt(p)| returns the square root of p.
    \end{frame}


    \begin{frame}[fragile]{Example: Quadratic Equation: Handling Edge-Case}
        Sometimes a quadratic equation may not have real-valued solutions.
        \begin{verbatim}
Start
Input a, b, c
If (b^2 - 4ac) < 0:
    Output: No real-valued soltions
Else:
    x1 = (-b + sqrt(b^2 - 4ac)) / 2a
    x2 = (-b - sqrt(b^2 - 4ac)) / 2a
    Output: x1, x2
EndIf
End
        \end{verbatim}
    \end{frame}


    \begin{frame}[fragile]{Example: Quadratic Equation: Handling Edge-Case (cont.)}
        Sometimes, there might be only one real-valued solution.
        \begin{verbatim}
Start
Input a, b, c
If (b^2 - 4ac) < 0:
    Output: No real-valued soltions
ElseIf (b^2 - 4ac) == 0:
    x = -b / 2a
    Output: x
Else:
    x1 = (-b + sqrt(b^2 - 4ac)) / 2a
    x2 = (-b - sqrt(b^2 - 4ac)) / 2a
    Output: x1, x2
EndIf
End
        \end{verbatim}
    \end{frame}


    \section{Exercises}

    \begin{frame}{Exercises}
    \begin{enumerate}
        \item Design an algorithm and flow chart to find the largest of the three numbers
        \item Develop pseudocode for computing the sum of the digits of a given integer
        \item Write an algorithm and pseudocode to check whether a number is prime
        \item Write a pseudocode for the Euclidean algorithm of finding GCD of two integers
        \item Write a pseudocode for finding the LCM of two integers
    \end{enumerate}
    \end{frame}

    \section*{Questions?}
\end{document}
