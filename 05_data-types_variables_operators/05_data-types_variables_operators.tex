% \documentclass[12pt, handout]{beamer}
\documentclass[12pt, aspectratio=169]{beamer}

\usetheme{moloch}
\molochset{block=fill}

\usepackage{fontspec}
\setsansfont[
    UprightFont = Inter-Light,          % Use light as the normal font weight
    BoldFont = Inter-SemiBold,           % Use semibold for \textbf
    ItalicFont = Inter-LightItalic,      % Light italic for \textit
    BoldItalicFont = Inter-SemiBoldItalic % Semibold italic for \textbf with \textit
]{Inter}[RawFeature={+ss04, +ss03, +dlig, +tnum}]
% Inter font stylistic sets:
% ss01: alternate digits for 3, 4, 6, 8
% ss02: for disambiguation (with zero) in places like "Ill" or "O0" etc
% ss04: for disambiguation (without zero) in places like "Ill" etc
% ss03: round comma, quation marks

\usepackage{upquote}
\usepackage{microtype}
\UseMicrotypeSet[protrusion]{basicmath}    % disable protrusion for tt fonts

\usepackage{amsmath}
\usepackage{amssymb}
\usepackage{unicode-math}
\setmathfont{Erewhon Math}[Scale=1.14]

\usepackage{hyperref}
\pdfstringdefDisableCommands{\def\translate#1{#1}}
\usepackage{bookmark}
\usepackage{url}

\usepackage{natbib}
\usepackage{appendixnumberbeamer}
\usepackage{enumerate}
% \usepackage{enumitem}    % it is giving the error: TeX capacity exceeded
% \usepackage{footnotehyper}
\usepackage{graphicx}
\usepackage{caption}
% \usepackage{subcaption}
\usepackage{booktabs}
\usepackage{makecell}
\usepackage{array}
\newcolumntype{H}{>{\setbox0=\hbox\bgroup\let\pm\relax}c<{\egroup}@{}}
% \newcolumntype{H}{>{\setbox0=\hbox\bgroup}c<{\egroup}}% <--- removed @{}
% https://tex.stackexchange.com/questions/567724/can-i-hide-a-table-column-with-the-s-type-from-siunitx
% https://tex.stackexchange.com/questions/414143/hide-column-without-adding-whitespace-to-table


\definecolor{airforceblue}{rgb}{0.36, 0.54, 0.66}
\hypersetup{
    colorlinks=true,
    linkcolor={mDarkTeal},    % this colour is defined by the moloch theme
    filecolor={Maroon},
    citecolor={airforceblue!120},
    urlcolor={airforceblue!140},
    pdfcreator={xelatex},
    bookmarksopen=true,    % Expand bookmarks in the PDF
    bookmarksnumbered=true % Include numbering in bookmarks
}

\bibliographystyle{apalike}

\let\oldcite=\cite
\renewcommand{\cite}[1]{\textcolor{airforceblue!120}{\oldcite{#1}}}
\let\oldcitet=\citet
\renewcommand{\citet}[1]{\textcolor{airforceblue!120}{\oldcitet{#1}}}
\let\oldcitep=\citep
\renewcommand{\citep}[1]{\textcolor{airforceblue!120}{\oldcitep{#1}}}


% \setlength{\leftmargini}{0em}
\setbeamercolor{page number in head/foot}{fg=gray}
\setbeamertemplate{footline}[frame number]
\setbeamertemplate{itemize items}[circle]
\setbeamertemplate{enumerate items}[circle]
\setbeamertemplate{sections/subsections in toc}[circle]
\setbeamertemplate{frametitle continuation}[from second][(cont.)]
\setbeamercovered{transparent}
\beamertemplatenavigationsymbolsempty


\AtBeginSubsection[]{
    {
        \begin{frame}[noframenumbering, plain]
            \subsectionpage
        \end{frame}
    }
}


\newcommand\Wider[2][4em]{%
    \makebox[\linewidth][c]{%
        \begin{minipage}{\dimexpr\textwidth+#1\relax}
            % \raggedright#2
            \centering#2
        \end{minipage}%
    }%
}

% \newenvironment{myitemize}{
%     \begin{itemize}
%         \vspace{1em}
%         \setlength{\itemsep}{0.7\baselineskip}
% }{
%         \vspace{1em}
%     \end{itemize}
% }

% \newenvironment{myenumerate}{
%     \begin{enumerate}
%         \vspace{1em}
%         \setlength{\itemsep}{0.7\baselineskip}
% }{
%         \vspace{1em}
%     \end{enumerate}
% }


\title{Data Types, Variables, and Operators in C}
\author{Md. Aminul Islam Shazid}
\date{}


\begin{document}
    {
		\setbeamertemplate{footline}{}    % NO FOOTLINE FOR THESE TWO FRAMES
		\addtocounter{framenumber}{-2}    % not counting the title page and the outline in frame numbers

		\begin{frame}
			\titlepage
		\end{frame}

		\begin{frame}{Outline}
            \vfill
			\tableofcontents[subsectionstyle=hide]
            \vfill
		\end{frame}
	}

    \section{Data Types and Variables}

    \begin{frame}{Basic Data Types in C}
        \begin{itemize}
            \item \textbf{int}: whole numbers (typically 4 bytes) \\
                \textit{Usage:} integer data, counters, loop indices
            \item \textbf{float}: single-precision decimals (\~{}6 digits) \\
                \textit{Usage:} decimal data
            \item \textbf{double}: double-precision decimals (\~{}15 digits) \\
                \textit{Usage:} precise calculations, finance
            \item \textbf{char}: single character (1 byte, ASCII) \\
                \textit{Usage:} characters, text handling
            \item \textbf{void}: represents no value \\
                \textit{Usage:} function return type, pointers
            \item \textbf{short, long, unsigned}: integer variants \\
                \textit{Usage:} memory optimization, large values
        \end{itemize}
    \end{frame}


    \begin{frame}{Variable Sizes and Precision}
        \begin{itemize}
            \item \textbf{Sizes vary by system/compiler}, but common values:
                \begin{itemize}
                    \item \texttt{char}: 1 byte
                    \item \texttt{short}: 2 bytes
                    \item \texttt{int}: 4 bytes
                    \item \texttt{long}: 4 or 8 bytes
                    \item \texttt{float}: 4 bytes (about 6 decimal digits)
                    \item \texttt{double}: 8 bytes (about 15 decimal digits)
                \end{itemize}
            \item Use \texttt{sizeof()} operator to check actual size
            \item Precision: \texttt{float} (single) vs. \texttt{double} (double precision)
        \end{itemize}
    \end{frame}


    \begin{frame}{Variable Definition and Declaration}
        \begin{itemize}
            \item Syntax: \texttt{data\_type variable\_name;}
            \item Initialization: \texttt{int x = 10;}
            \item Can also do: \texttt{int x; x = 10;}
            \item Scope:
                \begin{itemize}
                    \item Local: inside a function
                    \item Global: outside all functions
                \end{itemize}
            \item Constants:
                \begin{itemize}
                    \item \texttt{const int MAX = 100;}
                    \item \texttt{\#define PI 3.14}
                \end{itemize}
        \end{itemize}
    \end{frame}


    \begin{frame}{Type Casting in C}
        \begin{itemize}
            \item \textbf{Type casting} converts a variable from one data type to another
            \item \textbf{Implicit casting (type promotion):}
                \begin{itemize}
                    \item Done automatically by the compiler
                    \item Example: \texttt{int x = 5; double y = x;} \hfill \emph{(x promoted to double)}
                \end{itemize}
            \item \textbf{Explicit casting:}
                \begin{itemize}
                    \item Done by the programmer using cast operator
                    \item Syntax: \texttt{(type) expression}
                    \item Example: \texttt{double a = 5.7; int b = (int)a;} \hfill \emph{(b = 5)}
                \end{itemize}
            \item Use casting carefully: may cause data loss (e.g., truncation)
        \end{itemize}
    \end{frame}


    \begin{frame}{Variable Naming Rules in C}
        \begin{itemize}
            \item Must begin with a letter or underscore (\_)
            \item Can contain letters, digits, and underscores
            \item Case-sensitive: \texttt{value} and \texttt{Value} are different
            \item Cannot be a reserved keyword (\texttt{int}, \texttt{return}, etc.)
            \item Should be meaningful for readability (e.g., \texttt{total}, not \texttt{x1})
        \end{itemize}
    \end{frame}


    \section{Operators}

    \begin{frame}{Operators in C}
        \begin{itemize}
            \item \textbf{Arithmetic:} \texttt{+}, \texttt{-}, \texttt{*}, \texttt{/}, \texttt{\%} \\
                Perform basic mathematical operations
            \item \textbf{Relational:} \texttt{<}, \texttt{<=}, \texttt{>}, \texttt{>=}, \texttt{==}, \texttt{!=} \\
                Compare two values, result is either true (1) or false (0)
            \item \textbf{Logical:} \texttt{\&\&}, \texttt{||}, \texttt{!} \\
                Combine conditions: \texttt{\&\&} (AND), \texttt{||} (OR), \texttt{!} (NOT)
            \item \textbf{Assignment:} \texttt{=}, \texttt{+=}, \texttt{-=}, \texttt{*=}, \texttt{/=} \\
                Store values in variables or update them with shorthand forms
        \end{itemize}
    \end{frame}


    \begin{frame}{Prefix vs Postfix Operators}
        \begin{itemize}
            \item \textbf{Increment / Decrement operators:} \texttt{++}, \texttt{--}
            \item \textbf{Prefix form} (\texttt{++x}, \texttt{--x})
                \begin{itemize}
                    \item Variable is updated first, then used in the expression
                    \item Example: 
                        \begin{itemize}
                            \item \texttt{int x = 5;}
                            \item \texttt{int y = ++x;} \hfill \emph{(\texttt{x=6}, \texttt{y=6})}
                        \end{itemize}
                \end{itemize}
            \item \textbf{Postfix form} (\texttt{x++}, \texttt{x--})
                \begin{itemize}
                    \item Variable is used first, then updated
                    \item Example: 
                        \begin{itemize}
                            \item \texttt{int x = 5;}
                            \item \texttt{int y = x++;} \hfill \emph{(\texttt{x=6}, \texttt{y=5})}
                        \end{itemize}
                \end{itemize}
            \item Rule of thumb: prefix = ``increment before use", postfix = ``increment after use".
        \end{itemize}
    \end{frame}


    \begin{frame}{Truth Tables for Logical Operators}
        \begin{columns}

        \column{0.33\textwidth}
            \textbf{AND (\&\&)} \vspace{1em}

            \begin{tabular}{ccc}
                A & B & A \&\& B \\
                \hline
                0 & 0 & 0 \\
                0 & 1 & 0 \\
                1 & 0 & 0 \\
                1 & 1 & 1 \\
            \end{tabular}

        \column{0.33\textwidth}
            \textbf{OR (||)} \vspace{1em}

            \begin{tabular}{ccc}
                A & B & A || B \\
                \hline
                0 & 0 & 0 \\
                0 & 1 & 1 \\
                1 & 0 & 1 \\
                1 & 1 & 1 \\
            \end{tabular}

        \column{0.33\textwidth}
            \textbf{NOT (!)} \vspace{1em}

            \begin{tabular}{cc}
                A & !A \\
                \hline
                0 & 1 \\
                1 & 0 \\
            \end{tabular}

        \end{columns}
    \end{frame}


    \begin{frame}{Order of Evaluation and Precedence}
        Operators in C follow a precedence hierarchy.\\

        Examples (highest to lowest):
            \begin{itemize}
                \item \texttt{()}: Parentheses
                \item \texttt{*}, \texttt{/}, \texttt{\%}: Multiplication, Division, Modulus
                \item \texttt{+}, \texttt{-}: Addition, Subtraction
                \item \texttt{<}, \texttt{>}, \texttt{<=}, \texttt{>=}: Relational
                \item \texttt{==}, \texttt{!=}: Equality
                \item \texttt{\&\&}: Logical AND
                \item \texttt{||}: Logical OR
                \item \texttt{=}: Assignment (lowest)
            \end{itemize}
        Use parentheses \texttt{()} to make evaluation explicit.\\

        Example: \texttt{int x = 2 + 3 * 4;} → result is 14, not 20.
    \end{frame}


    \section{Input, Output (IO)}

    \begin{frame}{Formatted Output: printf()}
        \begin{itemize}
            \item Used to display output to the screen
            \item General form: \texttt{printf("format string", values);}
            \item Format specifiers:
                \begin{itemize}
                    \item \texttt{\%d} → integer
                    \item \texttt{\%f} → float/double
                    \item \texttt{\%c} → char
                    \item \texttt{\%s} → string
                \end{itemize}
            \item Example: \texttt{printf("Sum = \%d", x);}
        \end{itemize}
    \end{frame}


    \begin{frame}{Formatted Input: scanf()}
        \begin{itemize}
            \item Used to take input from the user
            \item General form: \texttt{scanf("format string", \&variables);}
            \item Format specifiers are the same as for \texttt{printf}
            \item Example: \texttt{scanf("\%d", \&x);}
        \end{itemize}
    \end{frame}


    \begin{frame}{Why use the ampersand sign (\&) in scanf()?}
        \begin{itemize}
            \item \texttt{scanf()} needs the \textbf{address of a variable} to store the input value
            \item The operator \texttt{\&} (“address-of”) provides that memory location
            \item Example:
                \begin{itemize}
                    \item \texttt{int x;}
                    \item \texttt{scanf("\%d", \&x);}  % correct
                    \item Without \texttt{\&}, the program will not know where to put the value
                \end{itemize}
            \item \textbf{Exception:} For strings (\texttt{\%s}), the variable itself is already an address, so no \texttt{\&} is needed
        \end{itemize}
    \end{frame}


    \section{Exercise}

    \begin{frame}{Exercise}
        \begin{itemize}
            \item Write a C program that demonstrates the basic arithmetic operations
            \item Write a C program that divides an 5 (integer) by 2 (integer), 5.0 (float) by 2 (integer), and 5 (integer) by 2.0 (float)
            \item Write a C program that checks whether a user-given number is odd or even, you can use the modulo (\%) operator
            \item Guess is the outputs:  \\
                \texttt{int x = 5; printf("\%d", x++);}  \\
                \texttt{int y = 5; printf("\%d", ++y);}
        \end{itemize}
    \end{frame}

    \section*{Questions?}

\end{document}
