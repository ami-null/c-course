% \documentclass[12pt, handout]{beamer}
\documentclass[12pt, aspectratio=169]{beamer}

\input{../header}
\usepackage{minted}
\renewcommand{\theFancyVerbLine}{\ttfamily{\small\oldstylenums{\arabic{FancyVerbLine}}}}
\setminted{
    frame=none,
    bgcolor=gray!20,
    % linenos, 
    % numbers=left,
    % numbersep=5pt
    breaklines=true,
    fontsize=\fontsize{12pt}{13pt}\selectfont
}



\title{Data Types, Variables, and Operators in C}
\author{Md. Aminul Islam Shazid}
\date{}


\begin{document}
    {
		\setbeamertemplate{footline}{}    % NO FOOTLINE FOR THESE TWO FRAMES
		\addtocounter{framenumber}{-2}    % not counting the title page and the outline in frame numbers

		\begin{frame}
			\titlepage
		\end{frame}

		\begin{frame}{Outline}
            \vfill
			\tableofcontents[subsectionstyle=hide]
            \vfill
		\end{frame}
	}

    \section{Data Types and Variables}

    \begin{frame}{Basic Data Types in C}
        \begin{itemize}
            \item \textbf{int}: whole numbers (typically 4 bytes) \\
                \textit{Usage:} integer data, counters, loop indices
            \item \textbf{float}: single-precision decimals \\
                \textit{Usage:} decimal data
            \item \textbf{double}: double-precision decimals\\
                \textit{Usage:} precise calculations, finance
            \item \textbf{char}: single character (1 byte, ASCII) \\
                \textit{Usage:} characters, text handling
            \item \textbf{void}: represents no value \\
                \textit{Usage:} function return type, pointers
            \item \textbf{short, long, unsigned}: integer variants \\
                \textit{Usage:} memory optimization, large values
        \end{itemize}
    \end{frame}


    \begin{frame}{Variable Sizes and Precision}
        \begin{itemize}
            \item \textbf{Sizes vary by system/compiler}, but common values:
                \begin{itemize}
                    \item \textttbg{char}: 1 byte
                    \item \textttbg{short}: 2 bytes
                    \item \textttbg{int}: 4 bytes
                    \item \textttbg{long}: 4 or 8 bytes
                    \item \textttbg{float}: 4 bytes (about 6 decimal digits)
                    \item \textttbg{double}: 8 bytes (about 15 decimal digits)
                \end{itemize}
            \item Use \textttbg{sizeof()} operator to check actual size
            \item Precision: \textttbg{float} (single) vs. \textttbg{double} (double precision)
        \end{itemize}
    \end{frame}


    \begin{frame}{Signed vs Unsigned Integers}
        \begin{itemize}
            \item \textbf{Signed integers} can store both positive and negative values
            \item \textbf{Unsigned integers} can store only non-negative values
            \item \textbf{Range difference:} For the same number of bytes, unsigned types can store roughly twice the positive range. 
                \textit{Example (32-bit):}\\
                \textttbg{int: -2,147,483,648 to 2,147,483,647}\\
                \textttbg{unsigned int: 0 to 4,294,967,295}
            \item \textbf{When to use unsigned:} When negative values are impossible or meaningless, e.g., array indices, sizes, counters, or bitwise operations
        \end{itemize}
    \end{frame}


    \begin{frame}{Variable Definition and Declaration}
        \begin{itemize}
            \item Syntax: \textttbg{data\_type variable\_name;}
            \item Initialization: \textttbg{int x = 10;}
            \item Can also do: \textttbg{int x; x = 10;}
            \item Scope:
                \begin{itemize}
                    \item Local: inside a function
                    \item Global: outside all functions
                \end{itemize}
            \item Constants:
                \begin{itemize}
                    \item \textttbg{const int MAX = 100;}
                    \item \textttbg{\#define PI 3.14}
                \end{itemize}
        \end{itemize}
    \end{frame}


    \begin{frame}{Type Casting in C}
        \begin{itemize}
            \item \textbf{Type casting} converts a variable from one data type to another
            \item \textbf{Implicit casting (type promotion):}
                \begin{itemize}
                    \item Done automatically by the compiler
                    \item Example: \textttbg{int x = 5; double y = x;} \hfill \emph{(x promoted to double)}
                \end{itemize}
            \item \textbf{Explicit casting:}
                \begin{itemize}
                    \item Done by the programmer using cast operator
                    \item Syntax: \textttbg{(type) expression}
                    \item Example: \textttbg{double a = 5.7; int b = (int)a;} \hfill \emph{(b = 5)}
                \end{itemize}
            \item Use casting carefully: may cause data loss (e.g., truncation)
        \end{itemize}
    \end{frame}


    \begin{frame}{Variable Naming Rules in C}
        \begin{itemize}
            \item Must begin with a letter or underscore (\_)
            \item Can contain letters, digits, and underscores
            \item Case-sensitive: \textttbg{value} and \textttbg{Value} are different
            \item Cannot be a reserved keyword (\textttbg{int}, \textttbg{return}, etc.)
            \item Should be meaningful for readability (e.g., \textttbg{total}, not \textttbg{x1})
        \end{itemize}
    \end{frame}


    \section{Operators}

    \begin{frame}{Operators in C}
        \begin{itemize}
            \item \textbf{Arithmetic:} \textttbg{+}, \textttbg{-}, \textttbg{*}, \textttbg{/}, \textttbg{\%} \\
                Perform basic mathematical operations
            \item \textbf{Relational:} \textttbg{<}, \textttbg{<=}, \textttbg{>}, \textttbg{>=}, \textttbg{==}, \textttbg{!=} \\
                Compare two values, result is either true (1) or false (0)
            \item \textbf{Logical:} \textttbg{\&\&}, \textttbg{||}, \textttbg{!} \\
                Combine conditions: \textttbg{\&\&} (AND), \textttbg{||} (OR), \textttbg{!} (NOT)
            \item \textbf{Assignment:} \textttbg{=}, \textttbg{+=}, \textttbg{-=}, \textttbg{*=}, \textttbg{/=} \\
                Store values in variables or update them with shorthand forms
            \item \textbf{Increment or decrement:} \textttbg{++}, \textttbg{--}\textbf{:} 
                \textttbg{++} increseases the value of a variable by 1, \textttbg{--} decreases the value of a variable by 1
        \end{itemize}
    \end{frame}


    \begin{frame}{Prefix vs Postfix Operators}
        \begin{itemize}
            \item \textbf{Increment / Decrement operators:} \textttbg{++}, \textttbg{--}
            \item \textbf{Prefix form} (\textttbg{++x}, \textttbg{--x})
                \begin{itemize}
                    \item Variable is updated first, then used in the expression
                    \item Example: 
                        \begin{itemize}
                            \item \textttbg{int x = 5;}
                            \item \textttbg{int y = ++x;} \hfill \emph{(Now: \texttt{x=6}, \texttt{y=6})}
                        \end{itemize}
                \end{itemize}
            \item \textbf{Postfix form} (\textttbg{x++}, \textttbg{x--})
                \begin{itemize}
                    \item Variable is used first, then updated
                    \item Example: 
                        \begin{itemize}
                            \item \textttbg{int x = 5;}
                            \item \textttbg{int y = x++;} \hfill \emph{(Now: \texttt{x=6}, \texttt{y=5})}
                        \end{itemize}
                \end{itemize}
            \item Rule of thumb: prefix: ``increment before use", postfix: ``increment after use".
        \end{itemize}
    \end{frame}


    \begin{frame}{Truth Tables for Logical Operators}
        \begin{columns}

        \column{0.33\textwidth}
            \textbf{AND (\&\&)} \vspace{1em}

            \begin{tabular}{ccc}
                A & B & A \&\& B \\
                \hline
                0 & 0 & 0 \\
                0 & 1 & 0 \\
                1 & 0 & 0 \\
                1 & 1 & 1 \\
            \end{tabular}

        \column{0.33\textwidth}
            \textbf{OR (||)} \vspace{1em}

            \begin{tabular}{ccc}
                A & B & A || B \\
                \hline
                0 & 0 & 0 \\
                0 & 1 & 1 \\
                1 & 0 & 1 \\
                1 & 1 & 1 \\
            \end{tabular}

        \column{0.33\textwidth}
            \textbf{NOT (!)} \vspace{1em}

            \begin{tabular}{cc}
                A & !A \\
                \hline
                0 & 1 \\
                1 & 0 \\
            \end{tabular}

        \end{columns}
    \end{frame}


    \begin{frame}{Order of Evaluation and Precedence}
        Operators in C follow a precedence hierarchy.

        \vspace{0.5em}

        Examples (highest to lowest):
            \begin{itemize}
                \item \textttbg{()}: Parentheses
                \item \textttbg{*}, \textttbg{/}, \textttbg{\%}: Multiplication, Division, Modulo
                \item \textttbg{+}, \textttbg{-}: Addition, Subtraction
                \item \textttbg{<}, \textttbg{>}, \textttbg{<=}, \textttbg{>=}: Relational
                \item \textttbg{==}, \textttbg{!=}: Equality
                \item \textttbg{\&\&}: Logical AND
                \item \textttbg{||}: Logical OR
                \item \textttbg{=}: Assignment (lowest)
            \end{itemize}
        Use parentheses \textttbg{()} to make evaluation explicit.

        \vspace{0.5em}

        Example: \mintinline{C}|int x = 2 + 3 * 4;| → result is 14, not 20.
    \end{frame}


    \section{Input, Output (IO)}

    \begin{frame}{Formatted Output: printf()}
        \begin{itemize}
            \item Used to display output to the screen
            \item Can display literal text as well as values of variables
            \item General form: \textttbg{printf("text to display", values);}
            \item To display values of variables, format specifiers are needed:
                \begin{itemize}
                    \item \textttbg{\%d} → integer
                    \item \textttbg{\%f} → float/double
                    \item \textttbg{\%c} → char
                    \item \textttbg{\%s} → string
                \end{itemize}
            \item Example: \textttbg{printf("Sum = \%d", x);}
            \item In the above, the value of \textttbg{x} is printed in place of \textttbg{\%d}
        \end{itemize}
    \end{frame}


    \begin{frame}{Formatted Input: scanf()}
        \begin{itemize}
            \item Used to take input from the user
            \item General form: \textttbg{scanf("format string", \&variables);}
            \item Format specifiers are the same as for \textttbg{printf()}
            \item Example: \textttbg{scanf("\%d", \&x);}
            \item In the above, the computer expects an integer value to be given as input by the user, then stores it in the variable \textttbg{x}
            \item Inputting multiple values: \textttbg{scanf("\%d \%d", \&x, \&y);}
        \end{itemize}
    \end{frame}


    \begin{frame}{Why use the ampersand sign (\&) in scanf()?}
        \begin{itemize}
            \item \textttbg{scanf()} needs the \textbf{address of a variable} to store the input value
            \item The operator \textttbg{\&} (“address-of”) provides that memory location
            \item Example:
                \begin{itemize}
                    \item \textttbg{int x;}
                    \item \textttbg{scanf("\%d", \&x);}  % correct
                    \item Without \textttbg{\&}, the program will not know where to put the value
                \end{itemize}
            \item \textbf{Exception:} For strings (\textttbg{\%s}), the variable itself already holds an address, so no \textttbg{\&} is needed
        \end{itemize}
    \end{frame}


    \section{Examples}

    \begin{frame}{Sum of Two Integers}
        \inputminted[
            linenos, 
        ]{c}{../code-examples/05_01_sum-two-integers.c}
    \end{frame}


    \begin{frame}{Sum of Two User-Given Integers}
        \inputminted[
            linenos, 
        ]{c}{../code-examples/05_02_sum-two-user-given-integers.c}
    \end{frame}


    \begin{frame}{Implicit Typecasting}
        \inputminted[
            linenos, 
        ]{c}{../code-examples/05_03_implicit-typecasting.c}
    \end{frame}


    \begin{frame}{Explicit Typecasting}
        \inputminted[
            linenos, 
        ]{c}{../code-examples/05_04_explicit-typecasting.c}
    \end{frame}


    \section{Exercise}

    \begin{frame}{Exercise}
        \begin{itemize}
            \item Write a C program that demonstrates the basic arithmetic operations
            \item Write a C program that divides an 5 (integer) by 2 (integer), 5.0 (float) by 2 (integer), and 5 (integer) by 2.0 (float)
            \item Guess the outputs:  \\
                \textttbg{int x = 5; printf("\%d", x++);}  \\
                \textttbg{int y = 5; printf("\%d", ++y);}
        \end{itemize}
    \end{frame}

    \section*{Questions?}

\end{document}
