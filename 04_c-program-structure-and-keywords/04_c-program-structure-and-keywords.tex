% \documentclass[12pt, handout]{beamer}
\documentclass[12pt, aspectratio=169]{beamer}

\input{../header}
\usepackage{minted}
\renewcommand{\theFancyVerbLine}{\ttfamily{\small\oldstylenums{\arabic{FancyVerbLine}}}}
\setminted{
    frame=none,
    bgcolor=gray!20,
    % linenos, 
    % numbers=left,
    % numbersep=5pt
    breaklines=true,
    fontsize=\fontsize{12pt}{13pt}\selectfont
}



\title{C Program Structure and Keywords}
\author{Md. Aminul Islam Shazid}
\date{}


\begin{document}
    {
		\setbeamertemplate{footline}{}    % NO FOOTLINE FOR THESE TWO FRAMES
		\addtocounter{framenumber}{-2}    % not counting the title page and the outline in frame numbers

		\begin{frame}
			\titlepage
		\end{frame}

		\begin{frame}{Outline}
            \vfill
			\tableofcontents[subsectionstyle=hide]
            \vfill
		\end{frame}
	}

    \section{Hello World}

    \begin{frame}{Hello World in C}
        \inputminted[
            linenos, 
            % firstline=1,
            % lastline=6,
            % firstnumber=1
        ]{c}{../code-examples/01_hello-world.c}
    \end{frame}


    % \begin{frame}{Greetings in C}
    %     \inputminted[
    %         linenos, 
    %     ]{c}{../code-examples/04_01_greetings.c}
    % \end{frame}


    \section{Program Structure}

    \begin{frame}{Preprocessor Macros}
    \begin{itemize}
        \item \textttbg{\#include}: include header files
        \item \textttbg{\#define}: define constants/macros
        \item Conditional macros: \textttbg{\#if}, \textttbg{\#ifdef}, \textttbg{\#ifndef}, \textttbg{\#else}, \textttbg{\#elif}, \textttbg{\#endif}: compile conditionally
        \item \textttbg{\#undef}: remove macro definitions
    \end{itemize}
    \end{frame}


    \begin{frame}{The \texttt{main} Function}
        \begin{itemize}
            \item Entry point of C programs
            \item Return type (\textttbg{int}) indicates exit status
            \item Function name: \textttbg{main}
            \item Parentheses for parameters (empty for now)
            \item Curly braces \{\} define the body
        \end{itemize}
    \end{frame}


    \begin{frame}{Defining Variables}
        \begin{itemize}
            \item Syntax: \mintinline{C}|data_type variable_name;|
            \item Must end with semicolon \textttbg{;}
            \item Variables must be declared before use
            \item Can assign values at declaration
        \end{itemize}
    \end{frame}


    \begin{frame}{Semicolons and Statements}
        \begin{itemize}
            \item Semicolon ends a statement. Two statements in one line:
            \begin{itemize}
                \item \mintinline{C}|int x = 10; y = x + 5;|
                \item Discouraged in practice
            \end{itemize}
            \item Multiple statements form the body of functions
            \item Common source of beginner errors
        \end{itemize}
    \end{frame}


    \begin{frame}{Calling Functions}
        \begin{itemize}
            \item Syntax: \mintinline{C}|functionName(arguments);|
            \item Parentheses hold arguments/parameters (can be empty)
            \item Arguments or parameters are inputs to a function
            \item Must match function definition
            \item More on functions in a dedicated lecture
        \end{itemize}
    \end{frame}


    % \begin{frame}{Return Values}
    %     \begin{itemize}
    %         \item Functions can return a value to the caller
    %         \item Syntax: \mintinline{C}|return expression;|
    %         \item In \mintinline{C}|main()|, \mintinline{C}|return 0;| indicates successful execution
    %         \item Non-zero return values often indicate an error
    %     \end{itemize}
    % \end{frame}


    \begin{frame}{Brackets in C}
        \begin{itemize}
            \item Parentheses \textttbg{()}: grouping expressions and function calls
            \item Curly braces \textttbg{\{\}}: define a block of code
            \item Square brackets \textttbg{[]}: array indexing
        \end{itemize}
    \end{frame}


    \begin{frame}[fragile]{Comments}
        \begin{itemize}
            \item Single-line, starts with double slashes (\textttbg{//})
            \item Multi-line, starts with \textttbg{/*} and ends with \textttbg{*/}, this syntax is often discouraged
            \item Instead begin the lines with double slashes for multiline comments
        \end{itemize}
        \begin{minted}[linenos]{c}
// this is a single line comment

/* this is
a multiline comment */

// this is also
// a multiline comment
        \end{minted}
    \end{frame}


    \begin{frame}{Escape Sequences}
        \begin{itemize}
            \item \textttbg{\textbackslash n}: newline
            \item \textttbg{\textbackslash t}: tab
            \item \textttbg{\textbackslash\textbackslash}: backslash
            \item \textttbg{\textbackslash"}: double quote
            \item \textttbg{\textbackslash'}: single quote
        \end{itemize}
    \end{frame}


    \begin{frame}[fragile]{Whitespace and Indentation}
        \begin{itemize}
            \item Whitespace is ignored (except in strings)
            \item Indentation improves readability
            \item Use a single tab or four spaces for one level of indentation
            \item Example:
        \end{itemize}
        \begin{minted}{c}
if(condition){
    statement;
}
        \end{minted}
    \end{frame}


    \begin{frame}{Coding Conventions}
        \begin{itemize}
            \item Meaningful variable names
            \item Consistent indentation
            \item Opening brace \textttbg{\{} on the same line as keyword
            \item Use comments for clarity
        \end{itemize}
    \end{frame}


    \section{Keywords in C}

    \begin{frame}{Data Types and Values}
        \begin{itemize}
            \item \textttbg{int}: integer type
            \item \textttbg{float}: single precision floating-point
            \item \textttbg{double}: double precision floating-point
            \item \textttbg{char}: single character
            \item \textttbg{void}: no return value / no data
            \item \textttbg{signed}, \textttbg{unsigned}: signed/unsigned integers
            \item \textttbg{short}, \textttbg{long}: specify integer size
        \end{itemize}
    \end{frame}


    \begin{frame}{Control Flow}
        \begin{itemize}
            \item \textttbg{if}, \textttbg{else}: conditional branching
            \item \textttbg{switch}, \textttbg{case}, \textttbg{default}: multi-way branching
            \item \textttbg{for}, \textttbg{while}, \textttbg{do}: loops
            \item \textttbg{break}: exit loop or switch
            \item \textttbg{continue}: skip current iteration
            \item \textttbg{goto}: jump to label (use sparingly)
            \item \textttbg{return}: exit function, optionally returning value
        \end{itemize}
    \end{frame}


    \begin{frame}{Structuring}
        \begin{itemize}
            \item \textttbg{struct}: group related variables
            \item \textttbg{union}: store different types in same memory
            \item \textttbg{enum}: named integer constants
            \item \textttbg{typedef}: define a type alias
        \end{itemize}
    \end{frame}


    \begin{frame}{Pointer and Address-of Operators}
        \begin{itemize}
            \item \textttbg{\textit{datatype *varname}}: declares a pointer
            \item \textttbg{\textit{*varname}}: dereferences a pointer to access the value
            \item \textttbg{\textit{\&varname}}: gives the memory address of a variable
            \item Example:
        \end{itemize}
        \inputminted[
            linenos, 
        ]{c}{../code-examples/04_02_pointer.c}
    \end{frame}


    \begin{frame}{Storage Classes}
        \begin{itemize}
            \item \textttbg{auto}: default local variable storage
            \item \textttbg{register}: hint to store variable in CPU register
            \item \textttbg{static}: preserve value between function calls
            \item \textttbg{extern}: variable defined elsewhere
        \end{itemize}
    \end{frame}


    \begin{frame}{Memory and Miscellaneous}
        \begin{itemize}
            \item \textttbg{const}: read-only variable
            \item \textttbg{volatile}: variable may change unexpectedly
            \item \textttbg{restrict}: pointer optimization hint
            \item \textttbg{inline}: suggest inline function expansion
            \item \textttbg{\_Atomic}: atomic variable access
            \item \textttbg{\_Thread\_local}: thread-local storage
            \item \textttbg{\_Noreturn}: function does not return
            \item \textttbg{sizeof}: size of object or type
            \item \textttbg{\_Alignas}, \textttbg{\_Alignof}: memory alignment
            \item \textttbg{\_Generic}: type-generic selection (C11)
        \end{itemize}
    \end{frame}


    \begin{frame}{Preprocessor Keywords}
        \begin{itemize}
            \item \textttbg{\#define}: define macro or constant
            \item \textttbg{\#include}: include header file
            \item \textttbg{\#if}: conditional compilation
            \item \textttbg{\#ifdef}: compile if macro defined
            \item \textttbg{\#ifndef}: compile if macro not defined
            \item \textttbg{\#else}, \textttbg{\#elif}: alternative conditions
            \item \textttbg{\#endif}: end conditional
            \item \textttbg{\#undef}: undefine macro
            \item \textttbg{\#line}: set line number for compiler messages
            \item \textttbg{\#error}: generate compilation error
            \item \textttbg{\#pragma}: compiler-specific instruction
        \end{itemize}
    \end{frame}


    \section{Exercise}

    \begin{frame}{Exercise}
        \begin{itemize}
            \item Modify the hello-world program to print something else.
            \item Modify the greetings example to print something else.
        \end{itemize}
    \end{frame}

    \section*{Questions?}

\end{document}
