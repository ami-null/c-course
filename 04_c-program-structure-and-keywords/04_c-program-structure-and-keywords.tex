% \documentclass[12pt, handout]{beamer}
\documentclass[12pt, aspectratio=169]{beamer}

\usetheme{moloch}
\molochset{block=fill}

\usepackage{fontspec}
\setsansfont[
    UprightFont = Inter-Light,          % Use light as the normal font weight
    BoldFont = Inter-SemiBold,           % Use semibold for \textbf
    ItalicFont = Inter-LightItalic,      % Light italic for \textit
    BoldItalicFont = Inter-SemiBoldItalic % Semibold italic for \textbf with \textit
]{Inter}[RawFeature={+ss04, +ss03, +dlig, +tnum}]
% Inter font stylistic sets:
% ss01: alternate digits for 3, 4, 6, 8
% ss02: for disambiguation (with zero) in places like "Ill" or "O0" etc
% ss04: for disambiguation (without zero) in places like "Ill" etc
% ss03: round comma, quation marks

\usepackage{upquote}
\usepackage{microtype}
\UseMicrotypeSet[protrusion]{basicmath}    % disable protrusion for tt fonts

\usepackage{amsmath}
\usepackage{amssymb}
\usepackage{unicode-math}
\setmathfont{Erewhon Math}[Scale=1.14]

\usepackage{hyperref}
\pdfstringdefDisableCommands{\def\translate#1{#1}}
\usepackage{bookmark}
\usepackage{url}

\usepackage{natbib}
\usepackage{appendixnumberbeamer}
\usepackage{enumerate}
% \usepackage{enumitem}    % it is giving the error: TeX capacity exceeded
% \usepackage{footnotehyper}
\usepackage{graphicx}
\usepackage{caption}
% \usepackage{subcaption}
\usepackage{booktabs}
\usepackage{makecell}
\usepackage{array}
\newcolumntype{H}{>{\setbox0=\hbox\bgroup\let\pm\relax}c<{\egroup}@{}}
% \newcolumntype{H}{>{\setbox0=\hbox\bgroup}c<{\egroup}}% <--- removed @{}
% https://tex.stackexchange.com/questions/567724/can-i-hide-a-table-column-with-the-s-type-from-siunitx
% https://tex.stackexchange.com/questions/414143/hide-column-without-adding-whitespace-to-table


\definecolor{airforceblue}{rgb}{0.36, 0.54, 0.66}
\hypersetup{
    colorlinks=true,
    linkcolor={mDarkTeal},    % this colour is defined by the moloch theme
    filecolor={Maroon},
    citecolor={airforceblue!120},
    urlcolor={airforceblue!140},
    pdfcreator={xelatex},
    bookmarksopen=true,    % Expand bookmarks in the PDF
    bookmarksnumbered=true % Include numbering in bookmarks
}

\bibliographystyle{apalike}

\let\oldcite=\cite
\renewcommand{\cite}[1]{\textcolor{airforceblue!120}{\oldcite{#1}}}
\let\oldcitet=\citet
\renewcommand{\citet}[1]{\textcolor{airforceblue!120}{\oldcitet{#1}}}
\let\oldcitep=\citep
\renewcommand{\citep}[1]{\textcolor{airforceblue!120}{\oldcitep{#1}}}


% \setlength{\leftmargini}{0em}
\setbeamercolor{page number in head/foot}{fg=gray}
\setbeamertemplate{footline}[frame number]
\setbeamertemplate{itemize items}[circle]
\setbeamertemplate{enumerate items}[circle]
\setbeamertemplate{sections/subsections in toc}[circle]
\setbeamertemplate{frametitle continuation}[from second][(cont.)]
\setbeamercovered{transparent}
\beamertemplatenavigationsymbolsempty


\AtBeginSubsection[]{
    {
        \begin{frame}[noframenumbering, plain]
            \subsectionpage
        \end{frame}
    }
}


\newcommand\Wider[2][4em]{%
    \makebox[\linewidth][c]{%
        \begin{minipage}{\dimexpr\textwidth+#1\relax}
            % \raggedright#2
            \centering#2
        \end{minipage}%
    }%
}

% \newenvironment{myitemize}{
%     \begin{itemize}
%         \vspace{1em}
%         \setlength{\itemsep}{0.7\baselineskip}
% }{
%         \vspace{1em}
%     \end{itemize}
% }

% \newenvironment{myenumerate}{
%     \begin{enumerate}
%         \vspace{1em}
%         \setlength{\itemsep}{0.7\baselineskip}
% }{
%         \vspace{1em}
%     \end{enumerate}
% }


\title{C Program Structure and Keywords}
\author{Md. Aminul Islam Shazid}
\date{}


\begin{document}
    {
		\setbeamertemplate{footline}{}    % NO FOOTLINE FOR THESE TWO FRAMES
		\addtocounter{framenumber}{-2}    % not counting the title page and the outline in frame numbers

		\begin{frame}
			\titlepage
		\end{frame}

		\begin{frame}{Outline}
            \vfill
			\tableofcontents[subsectionstyle=hide]
            \vfill
		\end{frame}
	}

    \section{Hello World}

    \begin{frame}{Hello World in C}
        \inputminted[
            frame=none,
            bgcolor=gray!20,
            linenos, 
            breaklines=true,
            fontsize=\fontsize{12pt}{13pt}\selectfont,
            % firstline=1,
            % lastline=6,
            % firstnumber=1
        ]{c}{../code-examples/01_hello-world.c}
    \end{frame}


    \begin{frame}{Greetings in C}
        \inputminted[
            frame=none,
            bgcolor=gray!20,
            linenos, 
            breaklines=true,
            fontsize=\fontsize{12pt}{13pt}\selectfont
        ]{c}{../code-examples/04_01_greetings.c}
    \end{frame}


    \section{Program Structure}

    \begin{frame}{Preprocessor Macros}
    \begin{itemize}
        \item \texttt{\#include}: include header files
        \item \texttt{\#define}: define constants/macros
        \item Conditional macros: \texttt{\#if}, \texttt{\#ifdef}, \texttt{\#ifndef}, \texttt{\#else}, \texttt{\#elif}, \texttt{\#endif}: compile conditionally
        \item \texttt{\#undef}: remove macro definitions
    \end{itemize}
    \end{frame}


    \begin{frame}{The \texttt{main} Function}
        \begin{itemize}
            \item Entry point of C programs
            \item Return type (\texttt{int}) indicates exit status
            \item Function name: \texttt{main}
            \item Parentheses for parameters (empty for now)
            \item Curly braces \{\} define the body
        \end{itemize}
    \end{frame}


    \begin{frame}{Defining Variables}
        \begin{itemize}
            \item Syntax: \texttt{type variable\_name;}
            \item Must end with semicolon \texttt{;}
            \item Variables must be declared before use
            \item Can assign values at declaration
        \end{itemize}
    \end{frame}


    \begin{frame}{Semicolons and Statements}
        \begin{itemize}
            \item Semicolon ends a statement. Two statements in one line:
            \begin{itemize}
                \item \texttt{int x = 10; y = x + 5;}
                \item Discouraged in practice
            \end{itemize}
            \item Multiple statements form the body of functions
            \item Common source of beginner errors
        \end{itemize}
    \end{frame}


    \begin{frame}{Calling Functions}
        \begin{itemize}
            \item Syntax: \texttt{functionName(arguments);}
            \item Parentheses hold arguments (can be empty)
            \item Must match function definition/prototype
        \end{itemize}
    \end{frame}


    \begin{frame}{Return Values}
        \begin{itemize}
            \item Functions can return a value to the caller
            \item Syntax: \texttt{return expression;}
            \item In \texttt{main()}, \texttt{return 0;} indicates successful execution
            \item Non-zero return values often indicate an error
        \end{itemize}
    \end{frame}


    \begin{frame}{Brackets in C}
        \begin{itemize}
            \item Parentheses \texttt{()}: grouping expressions and function calls
            \item Curly braces \texttt{\{\}}: define a block of code
            \item Square brackets \texttt{[]}: array indexing
        \end{itemize}
    \end{frame}


    \begin{frame}[fragile]{Comments}
        \begin{itemize}
            \item Single-line: \texttt{// comment}
            \item Multi-line:
        \end{itemize}
        \begin{verbatim}
/* this is
a multiline comment */
        \end{verbatim}
    \end{frame}


    \begin{frame}{Escape Sequences}
        \begin{itemize}
            \item \texttt{\textbackslash n}: newline
            \item \texttt{\textbackslash t}: tab
            \item \texttt{\textbackslash\textbackslash}: backslash
            \item \texttt{\textbackslash"}: double quote
            \item \texttt{\textbackslash'}: single quote
        \end{itemize}
    \end{frame}


    \begin{frame}[fragile]{Whitespace and Indentation}
        \begin{itemize}
            \item Whitespace is ignored (except in strings)
            \item Indentation improves readability
            \item Use a single tab or four spaces for one level of indentation
            \item Example:
        \end{itemize}
        \begin{verbatim}
if(condition){
    statement;
}
        \end{verbatim}
    \end{frame}


    \begin{frame}{Coding Conventions}
        \begin{itemize}
            \item Meaningful variable names
            \item Consistent indentation
            \item Opening brace \texttt{\{} on the same line as keyword
            \item Use comments for clarity
        \end{itemize}
    \end{frame}


    \section{Keywords in C}

    \begin{frame}{Data Types and Values}
        \begin{itemize}
            \item \texttt{int}: integer type
            \item \texttt{float}: single precision floating-point
            \item \texttt{double}: double precision floating-point
            \item \texttt{char}: single character
            \item \texttt{void}: no return value / no data
            \item \texttt{signed}, \texttt{unsigned}: signed/unsigned integers
            \item \texttt{short}, \texttt{long}: specify integer size
        \end{itemize}
    \end{frame}


    \begin{frame}{Control Flow}
        \begin{itemize}
            \item \texttt{if}, \texttt{else}: conditional branching
            \item \texttt{switch}, \texttt{case}, \texttt{default}: multi-way branching
            \item \texttt{for}, \texttt{while}, \texttt{do}: loops
            \item \texttt{break}: exit loop or switch
            \item \texttt{continue}: skip current iteration
            \item \texttt{goto}: jump to label (use sparingly)
            \item \texttt{return}: exit function, optionally returning value
        \end{itemize}
    \end{frame}


    \begin{frame}{Structuring}
        \begin{itemize}
            \item \texttt{struct}: group related variables
            \item \texttt{union}: store different types in same memory
            \item \texttt{enum}: named integer constants
            \item \texttt{typedef}: define a type alias
        \end{itemize}
    \end{frame}


    \begin{frame}{Pointer and Address-of Operators}
        \begin{itemize}
            \item \texttt{\textit{datatype *varname}}: declares a pointer
            \item \texttt{\textit{*varname}}: dereferences a pointer to access the value
            \item \texttt{\textit{\&varname}}: gives the memory address of a variable
            \item Example:
        \end{itemize}
        \inputminted[
            frame=none,
            bgcolor=gray!20,
            linenos, 
            breaklines=true,
            fontsize=\fontsize{12pt}{13pt}\selectfont
        ]{c}{../code-examples/04_02_pointer.c}
    \end{frame}


    \begin{frame}{Storage Classes}
        \begin{itemize}
            \item \texttt{auto}: default local variable storage
            \item \texttt{register}: hint to store variable in CPU register
            \item \texttt{static}: preserve value between function calls
            \item \texttt{extern}: variable defined elsewhere
        \end{itemize}
    \end{frame}


    \begin{frame}{Memory and Miscellaneous}
        \begin{itemize}
            \item \texttt{const}: read-only variable
            \item \texttt{volatile}: variable may change unexpectedly
            \item \texttt{restrict}: pointer optimization hint
            \item \texttt{inline}: suggest inline function expansion
            \item \texttt{\_Atomic}: atomic variable access
            \item \texttt{\_Thread\_local}: thread-local storage
            \item \texttt{\_Noreturn}: function does not return
            \item \texttt{sizeof}: size of object or type
            \item \texttt{\_Alignas}, \texttt{\_Alignof}: memory alignment
            \item \texttt{\_Generic}: type-generic selection (C11)
        \end{itemize}
    \end{frame}


    \begin{frame}{Preprocessor Keywords}
        \begin{itemize}
            \item \texttt{\#define}: define macro or constant
            \item \texttt{\#include}: include header file
            \item \texttt{\#if}: conditional compilation
            \item \texttt{\#ifdef}: compile if macro defined
            \item \texttt{\#ifndef}: compile if macro not defined
            \item \texttt{\#else}, \texttt{\#elif}: alternative conditions
            \item \texttt{\#endif}: end conditional
            \item \texttt{\#undef}: undefine macro
            \item \texttt{\#line}: set line number for compiler messages
            \item \texttt{\#error}: generate compilation error
            \item \texttt{\#pragma}: compiler-specific instruction
        \end{itemize}
    \end{frame}

    \section*{Questions?}

\end{document}
